%% Generated by Sphinx.
\def\sphinxdocclass{jupyterBook}
\documentclass[letterpaper,10pt,english]{jupyterBook}
\ifdefined\pdfpxdimen
   \let\sphinxpxdimen\pdfpxdimen\else\newdimen\sphinxpxdimen
\fi \sphinxpxdimen=.75bp\relax
\ifdefined\pdfimageresolution
    \pdfimageresolution= \numexpr \dimexpr1in\relax/\sphinxpxdimen\relax
\fi
%% let collapsible pdf bookmarks panel have high depth per default
\PassOptionsToPackage{bookmarksdepth=5}{hyperref}
%% turn off hyperref patch of \index as sphinx.xdy xindy module takes care of
%% suitable \hyperpage mark-up, working around hyperref-xindy incompatibility
\PassOptionsToPackage{hyperindex=false}{hyperref}
%% memoir class requires extra handling
\makeatletter\@ifclassloaded{memoir}
{\ifdefined\memhyperindexfalse\memhyperindexfalse\fi}{}\makeatother

\PassOptionsToPackage{warn}{textcomp}

\catcode`^^^^00a0\active\protected\def^^^^00a0{\leavevmode\nobreak\ }
\usepackage{cmap}
\usepackage{fontspec}
\defaultfontfeatures[\rmfamily,\sffamily,\ttfamily]{}
\usepackage{amsmath,amssymb,amstext}
\usepackage{polyglossia}
\setmainlanguage{english}



\setmainfont{FreeSerif}[
  Extension      = .otf,
  UprightFont    = *,
  ItalicFont     = *Italic,
  BoldFont       = *Bold,
  BoldItalicFont = *BoldItalic
]
\setsansfont{FreeSans}[
  Extension      = .otf,
  UprightFont    = *,
  ItalicFont     = *Oblique,
  BoldFont       = *Bold,
  BoldItalicFont = *BoldOblique,
]
\setmonofont{FreeMono}[
  Extension      = .otf,
  UprightFont    = *,
  ItalicFont     = *Oblique,
  BoldFont       = *Bold,
  BoldItalicFont = *BoldOblique,
]



\usepackage[Bjarne]{fncychap}
\usepackage[,numfigreset=1,mathnumfig]{sphinx}

\fvset{fontsize=\small}
\usepackage{geometry}


% Include hyperref last.
\usepackage{hyperref}
% Fix anchor placement for figures with captions.
\usepackage{hypcap}% it must be loaded after hyperref.
% Set up styles of URL: it should be placed after hyperref.
\urlstyle{same}

\addto\captionsenglish{\renewcommand{\contentsname}{Vettori}}

\usepackage{sphinxmessages}



        % Start of preamble defined in sphinx-jupyterbook-latex %
         \usepackage[Latin,Greek]{ucharclasses}
        \usepackage{unicode-math}
        % fixing title of the toc
        \addto\captionsenglish{\renewcommand{\contentsname}{Contents}}
        \hypersetup{
            pdfencoding=auto,
            psdextra
        }
        % End of preamble defined in sphinx-jupyterbook-latex %
        

\title{Matematica per le scuole superiori}
\date{Oct 29, 2024}
\release{}
\author{basics}
\newcommand{\sphinxlogo}{\vbox{}}
\renewcommand{\releasename}{}
\makeindex
\begin{document}

\pagestyle{empty}
\sphinxmaketitle
\pagestyle{plain}
\sphinxtableofcontents
\pagestyle{normal}
\phantomsection\label{\detokenize{intro::doc}}


\sphinxAtStartPar
Questo libro fa parte del materiale pensato per \sphinxhref{https://basics2022.github.io/bbooks-hs}{le scuole superiori}



\begin{DUlineblock}{0em}
\item[] \sphinxstylestrong{\Large Obiettivi}
\end{DUlineblock}
\begin{itemize}
\item {} 
\sphinxAtStartPar
Descrizione dello spazio e di oggetti (matematici e fisici) nello spazio

\item {} 
\sphinxAtStartPar
…

\end{itemize}

\begin{DUlineblock}{0em}
\item[] \sphinxstylestrong{\large Argomenti}
\end{DUlineblock}

\begin{DUlineblock}{0em}
\item[] \sphinxstylestrong{\large Argomenti principali}
\end{DUlineblock}

\sphinxAtStartPar
\sphinxstylestrong{Vettori.} Algebra e cenni di calcolo vettoriale in spazi euclidei (con coordinate cartesiane).

\sphinxAtStartPar
\sphinxstylestrong{Geometria analitica nel piano e nello spazio.}

\sphinxAtStartPar
\sphinxstylestrong{Calcolo infinitesimale.}

\sphinxAtStartPar
\sphinxstylestrong{Statistica.}

\begin{DUlineblock}{0em}
\item[] \sphinxstylestrong{\large Argomenti utili}
\end{DUlineblock}

\sphinxAtStartPar
Lista di argomenti utili, per trattare in maniera sufficientemente completa gli argomenti principali, anche se possono non essere svolti in maniera esaustiva: anche se le sezioni verranno scritte in maniera completa, si può pensare che queste siano solo sezioni di “appoggio” o di approfondimento personale per i più curiosi. Questi argomenti possono costituire il corpo vero e proprio della miscellanea di matematica per il triennio.

\sphinxAtStartPar
\sphinxstylestrong{Serie e successioni.}

\sphinxAtStartPar
\sphinxstylestrong{Algebra complessa e cenni di calcolo complesso.}

\sphinxAtStartPar
\sphinxstylestrong{Algebra lineare.}

\begin{DUlineblock}{0em}
\item[] \sphinxstylestrong{\large Pre\sphinxhyphen{}requisiti}
\end{DUlineblock}

\sphinxAtStartPar
\sphinxstylestrong{Insiemistica e logica.}

\sphinxAtStartPar
\sphinxstylestrong{Algebra sui numeri reali.}

\sphinxAtStartPar
\sphinxstylestrong{Geometria euclidea.}

\sphinxstepscope


\part{Vettori}

\sphinxstepscope

\begin{sphinxuseclass}{sd-container-fluid}
\begin{sphinxuseclass}{sd-sphinx-override}
\begin{sphinxuseclass}{sd-p-0}
\begin{sphinxuseclass}{sd-mt-2}
\begin{sphinxuseclass}{sd-mb-4}
\begin{sphinxuseclass}{sd-row}
\begin{sphinxuseclass}{sd-row-cols-2}
\begin{sphinxuseclass}{sd-gx-2}
\begin{sphinxuseclass}{sd-gy-1}
\begin{sphinxuseclass}{sd-col}
\begin{sphinxuseclass}{sd-d-flex-row}
\begin{sphinxuseclass}{sd-align-minor-center}
\begin{sphinxuseclass}{sd-container-fluid}
\begin{sphinxuseclass}{sd-sphinx-override}
\begin{sphinxuseclass}{sd-row}
\begin{sphinxuseclass}{sd-row-cols-2}
\begin{sphinxuseclass}{sd-row-cols-xs-2}
\begin{sphinxuseclass}{sd-row-cols-sm-3}
\begin{sphinxuseclass}{sd-row-cols-md-3}
\begin{sphinxuseclass}{sd-row-cols-lg-3}
\begin{sphinxuseclass}{sd-gx-3}
\begin{sphinxuseclass}{sd-gy-1}
\begin{sphinxuseclass}{sd-col}
\begin{sphinxuseclass}{sd-col-auto}
\begin{sphinxuseclass}{sd-d-flex-row}
\begin{sphinxuseclass}{sd-align-minor-center}
\sphinxAtStartPar
basics

\end{sphinxuseclass}
\end{sphinxuseclass}
\end{sphinxuseclass}
\end{sphinxuseclass}
\begin{sphinxuseclass}{sd-col}
\begin{sphinxuseclass}{sd-col-auto}
\begin{sphinxuseclass}{sd-d-flex-row}
\begin{sphinxuseclass}{sd-align-minor-center}
\sphinxAtStartPar
Oct 22, 2024

\end{sphinxuseclass}
\end{sphinxuseclass}
\end{sphinxuseclass}
\end{sphinxuseclass}
\begin{sphinxuseclass}{sd-col}
\begin{sphinxuseclass}{sd-col-auto}
\begin{sphinxuseclass}{sd-d-flex-row}
\begin{sphinxuseclass}{sd-align-minor-center}
\sphinxAtStartPar
1 min read

\end{sphinxuseclass}
\end{sphinxuseclass}
\end{sphinxuseclass}
\end{sphinxuseclass}
\end{sphinxuseclass}
\end{sphinxuseclass}
\end{sphinxuseclass}
\end{sphinxuseclass}
\end{sphinxuseclass}
\end{sphinxuseclass}
\end{sphinxuseclass}
\end{sphinxuseclass}
\end{sphinxuseclass}
\end{sphinxuseclass}
\end{sphinxuseclass}
\end{sphinxuseclass}
\end{sphinxuseclass}
\end{sphinxuseclass}
\end{sphinxuseclass}
\end{sphinxuseclass}
\end{sphinxuseclass}
\end{sphinxuseclass}
\end{sphinxuseclass}
\end{sphinxuseclass}
\end{sphinxuseclass}
\end{sphinxuseclass}

\chapter{Vettori}
\label{\detokenize{ch/vectors:vettori}}\label{\detokenize{ch/vectors:vectors-high-school}}\label{\detokenize{ch/vectors::doc}}
\sphinxAtStartPar
Mentre alcune grandezze possono essere rappresentate completamente da un numero (e un’unità di misura, se necessaria), molte altre grandezze richiedono una quantità maggiore di informazioni.

\sphinxAtStartPar
\sphinxstylestrong{I limiti dei numeri scalari.}
Ad esempio, le frasi:
\begin{itemize}
\item {} 
\sphinxAtStartPar
“preparo \(150 \ g\) di riso”

\item {} 
\sphinxAtStartPar
“oggi è una bella giornata di primavera con temperatura massima prevista di 25°C”

\item {} 
\sphinxAtStartPar
“il treno ha impiegato \(20\) minuti”

\end{itemize}

\sphinxAtStartPar
non lasciano alcun dubbio sulla quantità di riso, sulla previsione della temperatura, e sulla durata del viaggio.

\sphinxAtStartPar
D’altra parte, le frasi:
\begin{itemize}
\item {} 
\sphinxAtStartPar
“l’oasi si trova a \(5 \ km\) da qui”

\item {} 
\sphinxAtStartPar
“l’aereo è passato qui sopra a \(500 \ km/h\)”

\item {} 
\sphinxAtStartPar
“il ciclista ha ricevuto una spinta dai tifosi”

\end{itemize}

\sphinxAtStartPar
lasciano molti dubbi sulla situazione descritta. Se mi trovassi disidratato nel deserto mi farebbe piacere sapere che c’è un’oasi nei paraggi, ma mi farebbe ancora più piacere conoscere in quale direzione devo andare. La seconda frase non mi permette di decidere in quale direzione guardare per cercare l’aereo, se mi fossi perso il passaggio sopra la mia testa. La terza frase contiene ancora meno informazioni: ha ricevuto una spinta, ok, ma di quale entità? E in quale direzione? L’ha aiutato, l’ha frenato, l’ha sbilanciato?

\sphinxAtStartPar
La differenza della completezza dell’informazione è dovuta al fatto che:
\begin{itemize}
\item {} 
\sphinxAtStartPar
\sphinxstylestrong{massa}, \sphinxstylestrong{temperatura}, \sphinxstylestrong{tempo} sono \sphinxstylestrong{grandezze scalari}, che possono essere completamente rappresentate con un solo numero, con l’opportuna unità di misura,

\item {} 
\sphinxAtStartPar
\sphinxstylestrong{spostamento}, \sphinxstylestrong{velocità}, \sphinxstylestrong{forza} sono \sphinxstylestrong{grandezze vettoriali}, che per essere rappresentate completamente hanno bisogno di un’informazione su intensità e verso.

\end{itemize}

\sphinxAtStartPar
\sphinxstylestrong{Introduzione ai vettori.}
\begin{itemize}
\item {} 
\sphinxAtStartPar
\sphinxstylestrong{Geometria nello spazio euclideo.}

\item {} 
\sphinxAtStartPar
\sphinxstylestrong{Algebra vettoriale.}

\item {} 
\sphinxAtStartPar
\sphinxstylestrong{Cenni di calcolo vettoriale.}

\end{itemize}

\sphinxstepscope

\begin{sphinxuseclass}{sd-container-fluid}
\begin{sphinxuseclass}{sd-sphinx-override}
\begin{sphinxuseclass}{sd-p-0}
\begin{sphinxuseclass}{sd-mt-2}
\begin{sphinxuseclass}{sd-mb-4}
\begin{sphinxuseclass}{sd-row}
\begin{sphinxuseclass}{sd-row-cols-2}
\begin{sphinxuseclass}{sd-gx-2}
\begin{sphinxuseclass}{sd-gy-1}
\begin{sphinxuseclass}{sd-col}
\begin{sphinxuseclass}{sd-d-flex-row}
\begin{sphinxuseclass}{sd-align-minor-center}
\begin{sphinxuseclass}{sd-container-fluid}
\begin{sphinxuseclass}{sd-sphinx-override}
\begin{sphinxuseclass}{sd-row}
\begin{sphinxuseclass}{sd-row-cols-2}
\begin{sphinxuseclass}{sd-row-cols-xs-2}
\begin{sphinxuseclass}{sd-row-cols-sm-3}
\begin{sphinxuseclass}{sd-row-cols-md-3}
\begin{sphinxuseclass}{sd-row-cols-lg-3}
\begin{sphinxuseclass}{sd-gx-3}
\begin{sphinxuseclass}{sd-gy-1}
\begin{sphinxuseclass}{sd-col}
\begin{sphinxuseclass}{sd-col-auto}
\begin{sphinxuseclass}{sd-d-flex-row}
\begin{sphinxuseclass}{sd-align-minor-center}
\sphinxAtStartPar
basics

\end{sphinxuseclass}
\end{sphinxuseclass}
\end{sphinxuseclass}
\end{sphinxuseclass}
\begin{sphinxuseclass}{sd-col}
\begin{sphinxuseclass}{sd-col-auto}
\begin{sphinxuseclass}{sd-d-flex-row}
\begin{sphinxuseclass}{sd-align-minor-center}
\sphinxAtStartPar
Oct 22, 2024

\end{sphinxuseclass}
\end{sphinxuseclass}
\end{sphinxuseclass}
\end{sphinxuseclass}
\begin{sphinxuseclass}{sd-col}
\begin{sphinxuseclass}{sd-col-auto}
\begin{sphinxuseclass}{sd-d-flex-row}
\begin{sphinxuseclass}{sd-align-minor-center}
\sphinxAtStartPar
1 min read

\end{sphinxuseclass}
\end{sphinxuseclass}
\end{sphinxuseclass}
\end{sphinxuseclass}
\end{sphinxuseclass}
\end{sphinxuseclass}
\end{sphinxuseclass}
\end{sphinxuseclass}
\end{sphinxuseclass}
\end{sphinxuseclass}
\end{sphinxuseclass}
\end{sphinxuseclass}
\end{sphinxuseclass}
\end{sphinxuseclass}
\end{sphinxuseclass}
\end{sphinxuseclass}
\end{sphinxuseclass}
\end{sphinxuseclass}
\end{sphinxuseclass}
\end{sphinxuseclass}
\end{sphinxuseclass}
\end{sphinxuseclass}
\end{sphinxuseclass}
\end{sphinxuseclass}
\end{sphinxuseclass}
\end{sphinxuseclass}

\section{Spazio euclideo e vettori}
\label{\detokenize{ch/vectors/euclidean-space-hs:spazio-euclideo-e-vettori}}\label{\detokenize{ch/vectors/euclidean-space-hs:vectors-high-school-euclidean-geometry}}\label{\detokenize{ch/vectors/euclidean-space-hs::doc}}
\sphinxAtStartPar
\sphinxstylestrong{Definizione di uno spazio euclideo, \(E^n\).}
\begin{itemize}
\item {} 
\sphinxAtStartPar
\sphinxstylestrong{Idea.} Lo spazio euclideo è una formalizzazione dell’idea di spazio alla quale siamo abitutati nella \sphinxstylestrong{vita di tutti i giorni}, e ancora più di noi, l’unica idea di spazio alla quale erano abituati gli esseri umani fino al XIX secolo.
Pensiamo allo spazio di tutti i giorni come uno \sphinxstylestrong{spazio assolto} e \sphinxstylestrong{omogeneo} (\sphinxstylestrong{todo} è corretto attribuire il concetto di omogeneità allo spazio?), nel senso che la misura di distanze e di angoli, le ralazioni di congruenza non variano se vengono valutate in diversi punti dello spazio (\sphinxstylestrong{todo} migiorare spiegazione; aggiungere immagini?). Ad esempio, lo spazio euclideo di due dimensioni è quello descritto dai 5 postulati della geometria di Euclide, \sphinxstylestrong{quinto postulato} incluso.

\item {} 
\sphinxAtStartPar
Gli elementi dell’insieme \(E^n\) vengono chiamati punti.

\item {} 
\sphinxAtStartPar
Tra due punti \(P\), \(Q\) è possibile tracciare uno e un solo segmento orientato, \(\overrightarrow{QP} = P - Q\), definito \sphinxstylestrong{vettore euclideo}, o \sphinxstylestrong{vettore spaziale}

\item {} 
\sphinxAtStartPar
Operazioni:
\begin{itemize}
\item {} 
\sphinxAtStartPar
somma vettoriale:
\begin{equation*}
\begin{split}\begin{aligned}
      \overrightarrow{QP} & = \overrightarrow{QR} + \overrightarrow{RP} \\
     (P- Q) & = ( R - Q ) + ( P - R)
    \end{aligned}\end{split}
\end{equation*}
\item {} 
\sphinxAtStartPar
moltiplicazione per uno scalare: \sphinxstylestrong{todo}

\end{itemize}

\item {} 
\sphinxAtStartPar
Spazio vettoriale euclideo: definizione di un punto come origine

\item {} 
\sphinxAtStartPar
Base vettoriale

\item {} 
\sphinxAtStartPar
Norma, prodotto interno, prodotto vettoriale nello spazio euclideo 3\sphinxhyphen{}dimensionale.
\begin{itemize}
\item {} 
\sphinxAtStartPar
\sphinxstylestrong{todo.} introdurre le definizioni di \sphinxstylestrong{spazio metrico con prodotto interno}

\end{itemize}

\item {} 
\sphinxAtStartPar
\sphinxstylestrong{Invarianza}
\begin{itemize}
\item {} 
\sphinxAtStartPar
carattere assoluto dei vettori, che li rendono gli strumenti fondamentali per la formulazione di teorie che non dipendono dall’osservatore o da scelte arbitrarie

\item {} 
\sphinxAtStartPar
invarianza di:
\begin{itemize}
\item {} 
\sphinxAtStartPar
vettori

\item {} 
\sphinxAtStartPar
operazioni sui vettori

\end{itemize}

\end{itemize}

\end{itemize}

\sphinxAtStartPar
\sphinxstylestrong{Confronto tra}:
\begin{itemize}
\item {} 
\sphinxAtStartPar
\sphinxstylestrong{Spazi vettoriali e basi.}

\item {} 
\sphinxAtStartPar
\sphinxstylestrong{Spazi e sistemi di coordinate.}
\begin{itemize}
\item {} 
\sphinxAtStartPar
Esempi: diversi sistemi di coordinate (2D: cartesiane, cartesiane “ruotate”, “lineari”, polari,…; 3D: cartesiane, cilindriche, sferiche,…)

\item {} 
\sphinxAtStartPar
Introduzione al calcolo vettoriale

\end{itemize}

\end{itemize}

\sphinxstepscope

\begin{sphinxuseclass}{sd-container-fluid}
\begin{sphinxuseclass}{sd-sphinx-override}
\begin{sphinxuseclass}{sd-p-0}
\begin{sphinxuseclass}{sd-mt-2}
\begin{sphinxuseclass}{sd-mb-4}
\begin{sphinxuseclass}{sd-row}
\begin{sphinxuseclass}{sd-row-cols-2}
\begin{sphinxuseclass}{sd-gx-2}
\begin{sphinxuseclass}{sd-gy-1}
\begin{sphinxuseclass}{sd-col}
\begin{sphinxuseclass}{sd-d-flex-row}
\begin{sphinxuseclass}{sd-align-minor-center}
\begin{sphinxuseclass}{sd-container-fluid}
\begin{sphinxuseclass}{sd-sphinx-override}
\begin{sphinxuseclass}{sd-row}
\begin{sphinxuseclass}{sd-row-cols-2}
\begin{sphinxuseclass}{sd-row-cols-xs-2}
\begin{sphinxuseclass}{sd-row-cols-sm-3}
\begin{sphinxuseclass}{sd-row-cols-md-3}
\begin{sphinxuseclass}{sd-row-cols-lg-3}
\begin{sphinxuseclass}{sd-gx-3}
\begin{sphinxuseclass}{sd-gy-1}
\begin{sphinxuseclass}{sd-col}
\begin{sphinxuseclass}{sd-col-auto}
\begin{sphinxuseclass}{sd-d-flex-row}
\begin{sphinxuseclass}{sd-align-minor-center}
\sphinxAtStartPar
basics

\end{sphinxuseclass}
\end{sphinxuseclass}
\end{sphinxuseclass}
\end{sphinxuseclass}
\begin{sphinxuseclass}{sd-col}
\begin{sphinxuseclass}{sd-col-auto}
\begin{sphinxuseclass}{sd-d-flex-row}
\begin{sphinxuseclass}{sd-align-minor-center}
\sphinxAtStartPar
Oct 22, 2024

\end{sphinxuseclass}
\end{sphinxuseclass}
\end{sphinxuseclass}
\end{sphinxuseclass}
\begin{sphinxuseclass}{sd-col}
\begin{sphinxuseclass}{sd-col-auto}
\begin{sphinxuseclass}{sd-d-flex-row}
\begin{sphinxuseclass}{sd-align-minor-center}
\sphinxAtStartPar
1 min read

\end{sphinxuseclass}
\end{sphinxuseclass}
\end{sphinxuseclass}
\end{sphinxuseclass}
\end{sphinxuseclass}
\end{sphinxuseclass}
\end{sphinxuseclass}
\end{sphinxuseclass}
\end{sphinxuseclass}
\end{sphinxuseclass}
\end{sphinxuseclass}
\end{sphinxuseclass}
\end{sphinxuseclass}
\end{sphinxuseclass}
\end{sphinxuseclass}
\end{sphinxuseclass}
\end{sphinxuseclass}
\end{sphinxuseclass}
\end{sphinxuseclass}
\end{sphinxuseclass}
\end{sphinxuseclass}
\end{sphinxuseclass}
\end{sphinxuseclass}
\end{sphinxuseclass}
\end{sphinxuseclass}
\end{sphinxuseclass}

\section{Algebra vettoriale}
\label{\detokenize{ch/vectors/algebra-hs:algebra-vettoriale}}\label{\detokenize{ch/vectors/algebra-hs:vectors-high-school-algebra-hs}}\label{\detokenize{ch/vectors/algebra-hs::doc}}
\sphinxAtStartPar
\sphinxstylestrong{Definizione.} Uno spazio vettoriale è una struttura algebrica formata da:
\begin{itemize}
\item {} 
\sphinxAtStartPar
un insieme \(V\), i cui elementi sono chiamati \sphinxstylestrong{vettori}

\item {} 
\sphinxAtStartPar
un campo \(K\) (di solito quello dei numeri reali \(\mathbb{R}\) o complessi \(\mathbb{C}\)), i cui elementi sono chiamati \sphinxstylestrong{scalari}

\item {} 
\sphinxAtStartPar
due operazioni chiuse rispetto all’insieme \(V\) chiamate:
\begin{itemize}
\item {} 
\sphinxAtStartPar
somma vettoriale

\item {} 
\sphinxAtStartPar
moltiplicazione per uno scalare,
che soddisfano determinate proprietà che verranno elencate in seguito.

\end{itemize}

\end{itemize}

\sphinxAtStartPar
Un’operazione è chiusa rispetto a un’insieme, se il risultato delle operazioni è un elemento dell’insieme.

\sphinxAtStartPar
Nel seguito del capitolo verranno considerati solo campi vettoriali definiti sui numeri reali, per i quali \(\mathscr{K} = \mathbb{R}\).

\sphinxAtStartPar
\sphinxstylestrong{Operazioni sui vettori: definizione di spazio vettoriale.}
\begin{itemize}
\item {} 
\sphinxAtStartPar
La \sphinxstylestrong{somma} tra due vettori \(\mathbf{v}\), \(\mathbf{w} \, \in V\) è il vettore

\end{itemize}
\begin{equation*}
\begin{split}\vec{v} + \vec{w} \in \mathscr{V}\end{split}
\end{equation*}\begin{itemize}
\item {} 
\sphinxAtStartPar
La \sphinxstylestrong{moltiplicazione per uno scalare} di un vettore \(\mathbf{v} \in \mathscr{V}\) per uno scalare \(\alpha \in \mathbb{R}\) è il vettore

\end{itemize}
\begin{equation*}
\begin{split}\alpha \vec{v} \in \mathscr{V}\end{split}
\end{equation*}
\sphinxAtStartPar
Queste due operazioni devono soddisfare le seguenti proprietà:
\begin{itemize}
\item {} 
\sphinxAtStartPar
proprietà commutativa della somma

\end{itemize}
\begin{equation*}
\begin{split}\mathbf{u} + \mathbf{v} = \mathbf{v} + \mathbf{u} \qquad \forall \mathbf{u}, \mathbf{v} \in V\end{split}
\end{equation*}\begin{itemize}
\item {} 
\sphinxAtStartPar
proprietà associativa della somma

\end{itemize}
\begin{equation*}
\begin{split}(\mathbf{u} + \mathbf{v}) + \mathbf{w} = \mathbf{u} + ( \mathbf{v} + \mathbf{w} ) \qquad \forall \mathbf{u}, \mathbf{v}, mathbf{w} \in V\end{split}
\end{equation*}\begin{itemize}
\item {} 
\sphinxAtStartPar
esistenza dell’elemento neutro della somma

\end{itemize}
\begin{equation*}
\begin{split}\exists \mathbf{0}_V \in V \qquad s.t. \qquad \mathbf{u} + \mathbf{0}_V = \mathbf{u} \qquad \forall \mathbf{u} \in V\end{split}
\end{equation*}\begin{itemize}
\item {} 
\sphinxAtStartPar
esistenza dell’elemento inverso della somma

\end{itemize}
\begin{equation*}
\begin{split}\forall \mathbf{u} \in V \exists \mathbf{u}' \in V \qquad s.t. \qquad \mathbf{u} + \mathbf{u}' = \mathbf{0}\end{split}
\end{equation*}\begin{itemize}
\item {} 
\sphinxAtStartPar
proprietà associativa del prodotto scalare

\end{itemize}
\begin{equation*}
\begin{split}(\alpha \beta) \mathbf{u} = \alpha ( \beta \mathbf{u} ) \forall \alpha, \beta \in K, \ \mathbf{u} \in V\end{split}
\end{equation*}\begin{itemize}
\item {} 
\sphinxAtStartPar
esistenza dell’elemento neutro della moltiplicazione per uno scalare

\end{itemize}
\begin{equation*}
\begin{split}\exists 1 \in K \qquad s.t. \qquad 1 \, \mathbf{u} = \mathbf{u} \quad \forall \mathbf{u} \in V\end{split}
\end{equation*}\begin{itemize}
\item {} 
\sphinxAtStartPar
proprietà distributiva della moltiplicazione per uno scalare

\end{itemize}
\begin{equation*}
\begin{split}(\alpha + \beta) \mathbf{u} = \alpha \mathbf{u} + \beta \mathbf{u}\end{split}
\end{equation*}\begin{equation*}
\begin{split}\alpha (\mathbf{u} + \mathbf{v}) = \alpha \mathbf{u} + \alpha \mathbf{v} \end{split}
\end{equation*}
\sphinxAtStartPar
\sphinxstylestrong{Base di uno spazio} In uno spazio vettoriale, ogni vettore può essere rappresentato come una combinazione lineare di un insieme di vettori dello spazio, opportunamente scelti. Il numero minimo di questi vettori è definita come dimensione dello spazio vettoriale.

\sphinxAtStartPar
\sphinxstylestrong{Prodotto scalare \sphinxhyphen{} o prodotto interno.}
\begin{equation*}
\begin{split}\vec{v} \cdot \vec{w}: \mathscr{V} \times \mathscr{V} \rightarrow \mathbb{R}\end{split}
\end{equation*}\begin{equation*}
\begin{split}\vec{v} \cdot \vec{w} = |\vec{v}| |\vec{w}| \cos \theta_{\vec{v} \vec{w}}\end{split}
\end{equation*}
\sphinxAtStartPar
\sphinxstylestrong{Prodotto vettoriale.}
\begin{equation*}
\begin{split}\vec{v} \times \vec{w}: \mathscr{V} \times \mathscr{V} \rightarrow \mathscr{V}\end{split}
\end{equation*}\begin{equation*}
\begin{split}\vec{v} \times \vec{w} = \hat{k} |\vec{v}| |\vec{w}| \sin \theta_{\vec{v} \vec{w}}\end{split}
\end{equation*}
\sphinxAtStartPar
\sphinxstylestrong{Base cartesiana.} \(\{ \hat{e}_i \}_{i=1:n^d}\), \(\{ \hat{x}, \hat{y}, \hat{z} \}\),
\begin{equation*}
\begin{split}\begin{aligned}
  \hat{x} \cdot  \hat{x} & = \hat{y} \cdot  \hat{y} = \hat{z} \cdot  \hat{z} = 1 \\
  \hat{x} \cdot  \hat{y} & = \hat{y} \cdot  \hat{z} = \hat{z} \cdot  \hat{x} = 0
\end{aligned}\end{split}
\end{equation*}
\sphinxAtStartPar
e usando il prodotto vettore per definire l’orientazione dei 3 vettori,
\begin{equation*}
\begin{split}\begin{aligned}
  \hat{x} \times \hat{y} & = \hat{z} \\
  \hat{y} \times \hat{z} & = \hat{x} \\
  \hat{z} \times \hat{x} & = \hat{y} \\
\end{aligned}\end{split}
\end{equation*}
\sphinxAtStartPar
Un vettore può essere scritto in una base cartesiana come
\begin{equation*}
\begin{split}\vec{v} = v_x \hat{x} + v_y \hat{y} + v_z \hat{z}\end{split}
\end{equation*}
\sphinxAtStartPar
Una coordinata del vettore è semplicemente
\begin{equation*}
\begin{split}v_x = \hat{x} \cdot \vec{v} \ .\end{split}
\end{equation*}
\sphinxAtStartPar
Somma di vettori e moltiplicazione per uno scalare
\begin{equation*}
\begin{split}\begin{aligned}
   \vec{v} + \vec{w} & =   (v_x \hat{x} + v_y \hat{y} + v_z \hat{z}) + (w_x \hat{x} + w_y \hat{y} + w_z \hat{z}) = \\
                     & =   (v_x + w_x) \hat{x} + (v_y + w_y) \hat{y} + (v_z + w_z) \hat{z} \\ \\
 a \vec{v}           & = a (v_x \hat{x} + v_y \hat{y} + v_z \hat{z}) = \\
                     & =   ( a v_x ) \hat{x} + ( a v_y ) \hat{y} + ( a v_z ) \hat{z}
\end{aligned}\end{split}
\end{equation*}
\sphinxAtStartPar
Il prodotto scalare diventa quindi
\begin{equation*}
\begin{split}\begin{aligned}
  \vec{v} \cdot \vec{w}
  & = (v_x \hat{x} + v_y \hat{y} + v_z \hat{z}) \cdot (w_x \hat{x} + w_y \hat{y} + w_z \hat{z}) = \\
  & = v_x w_x + v_y w_y + v_z w_z
\end{aligned}\end{split}
\end{equation*}
\sphinxAtStartPar
mentre il prodotto vettoriale
\begin{equation*}
\begin{split}\begin{aligned}
  \vec{v} \times \vec{w}
  & = (v_x \hat{x} + v_y \hat{y} + v_z \hat{z}) \times (w_x \hat{x} + w_y \hat{y} + w_z \hat{z}) = \\
  & = (v_y w_z - v_z w_y) \hat{x} + (v_z w_x - v_x w_z) \hat{y} + (v_x w_y - v_y w_x) \hat{z} = \\
\end{aligned}\end{split}
\end{equation*}

\subsection{Calcolo vettoriale in spazi euclidei, usando coordinate cartesiane}
\label{\detokenize{ch/vectors/algebra-hs:calcolo-vettoriale-in-spazi-euclidei-usando-coordinate-cartesiane}}
\sphinxstepscope


\part{Geometria analitica}

\sphinxstepscope


\chapter{Geometria analitica}
\label{\detokenize{ch/analytic_geometry:geometria-analitica}}\label{\detokenize{ch/analytic_geometry:geometry-analytic}}\label{\detokenize{ch/analytic_geometry::doc}}\begin{itemize}
\item {} 
\sphinxAtStartPar
La geometria analitica si occupa dello stuido delle figure geometriche nello spazio tramite l’uso di \sphinxstylestrong{sistemi di coordinate}: la scelta può arbitaria, spesso guidata da criteri di “comodità”; i risultati sono indipendenti dalla scelta

\item {} 
\sphinxAtStartPar
L’utilizzo di un sistema di coordinate per la descrizione dello spazio produce un legame tra la \sphinxstylestrong{geometria} e l’\sphinxstylestrong{algebra}:
\begin{itemize}
\item {} 
\sphinxAtStartPar
da un lato, le entità geometriche possono essere rappresentate con funzioni, equazioni e/o disequazioni che coinvolgono le coordinate;

\item {} 
\sphinxAtStartPar
dall’altro, ai problemi algebrici si può dare un’interpretazione geometrica;

\end{itemize}

\item {} 
\sphinxAtStartPar
Nel 1637 Cartesio formalizzò le basi della geometria analitica, o geometria cartesiana, nel libro \sphinxstyleemphasis{Geometria}, introdotto dal suo più famoso \sphinxstyleemphasis{Discorso sul metodo}.

\item {} 
\sphinxAtStartPar
Il lavoro di Cartesio fornisce strumenti fondamentali usati nella seconda metà del XVII secolo da Newton e Leibniz per sviluppare il calcolo infinitesimale, e in contemporaneanmente la meccanica razionale di Newton.

\end{itemize}

\sphinxAtStartPar
\sphinxstylestrong{Argomenti.}
\begin{itemize}
\item {} 
\sphinxAtStartPar
\sphinxstylestrong{Spazi euclidei}

\item {} 
\sphinxAtStartPar
\sphinxstylestrong{Geometria nel piano \sphinxhyphen{} spazio euclideo 2D, \(E^2\)}
\begin{itemize}
\item {} 
\sphinxAtStartPar
{\hyperref[\detokenize{ch/analytic_geometry/analytic_geometry_2d/coordinates:geometry-analytic-2d-coordinates}]{\sphinxcrossref{\DUrole{std,std-ref}{Sistemi di coordinate}}}} \sphinxstyleemphasis{Cartesiane e polari; trasformazione tra sistemi di coordinate: polari e cartesiane; cartesiano\sphinxhyphen{}cartesiano: traslazione e rotazione}

\item {} 
\sphinxAtStartPar
{\hyperref[\detokenize{ch/analytic_geometry/analytic_geometry_2d/points:geometry-analytic-2d-points}]{\sphinxcrossref{\DUrole{std,std-ref}{Punti}}}}

\item {} 
\sphinxAtStartPar
{\hyperref[\detokenize{ch/analytic_geometry/analytic_geometry_2d/lines:geometry-analytic-2d-lines}]{\sphinxcrossref{\DUrole{std,std-ref}{Rette}}}}

\item {} 
\sphinxAtStartPar
{\hyperref[\detokenize{ch/analytic_geometry/analytic_geometry_2d/conics:geometry-analytic-2d-conics}]{\sphinxcrossref{\DUrole{std,std-ref}{Coniche}}}} \sphinxstylestrong{todo} \sphinxstyleemphasis{parabola, ellisse, iperbole: def, caratteristiche, descrizione in coord. cartesiane e polari; riferimento a gravitazione in meccanica classica}

\end{itemize}

\item {} 
\sphinxAtStartPar
\sphinxstylestrong{Geometria nello spazio euclideo 3D, \(E^3\)}
\begin{itemize}
\item {} 
\sphinxAtStartPar
Sistemi di coordinate

\item {} 
\sphinxAtStartPar
Punti

\item {} 
\sphinxAtStartPar
Rette

\item {} 
\sphinxAtStartPar
Piani

\item {} 
\sphinxAtStartPar
Cono e rivisitazione delle coniche

\item {} 
\sphinxAtStartPar
Superfici quadratiche

\end{itemize}

\end{itemize}



\sphinxstepscope

\begin{sphinxuseclass}{sd-container-fluid}
\begin{sphinxuseclass}{sd-sphinx-override}
\begin{sphinxuseclass}{sd-p-0}
\begin{sphinxuseclass}{sd-mt-2}
\begin{sphinxuseclass}{sd-mb-4}
\begin{sphinxuseclass}{sd-row}
\begin{sphinxuseclass}{sd-row-cols-2}
\begin{sphinxuseclass}{sd-gx-2}
\begin{sphinxuseclass}{sd-gy-1}
\begin{sphinxuseclass}{sd-col}
\begin{sphinxuseclass}{sd-d-flex-row}
\begin{sphinxuseclass}{sd-align-minor-center}
\begin{sphinxuseclass}{sd-container-fluid}
\begin{sphinxuseclass}{sd-sphinx-override}
\begin{sphinxuseclass}{sd-row}
\begin{sphinxuseclass}{sd-row-cols-2}
\begin{sphinxuseclass}{sd-row-cols-xs-2}
\begin{sphinxuseclass}{sd-row-cols-sm-3}
\begin{sphinxuseclass}{sd-row-cols-md-3}
\begin{sphinxuseclass}{sd-row-cols-lg-3}
\begin{sphinxuseclass}{sd-gx-3}
\begin{sphinxuseclass}{sd-gy-1}
\begin{sphinxuseclass}{sd-col}
\begin{sphinxuseclass}{sd-col-auto}
\begin{sphinxuseclass}{sd-d-flex-row}
\begin{sphinxuseclass}{sd-align-minor-center}
\sphinxAtStartPar
basics

\end{sphinxuseclass}
\end{sphinxuseclass}
\end{sphinxuseclass}
\end{sphinxuseclass}
\begin{sphinxuseclass}{sd-col}
\begin{sphinxuseclass}{sd-col-auto}
\begin{sphinxuseclass}{sd-d-flex-row}
\begin{sphinxuseclass}{sd-align-minor-center}
\sphinxAtStartPar
Oct 22, 2024

\end{sphinxuseclass}
\end{sphinxuseclass}
\end{sphinxuseclass}
\end{sphinxuseclass}
\begin{sphinxuseclass}{sd-col}
\begin{sphinxuseclass}{sd-col-auto}
\begin{sphinxuseclass}{sd-d-flex-row}
\begin{sphinxuseclass}{sd-align-minor-center}
\sphinxAtStartPar
2 min read

\end{sphinxuseclass}
\end{sphinxuseclass}
\end{sphinxuseclass}
\end{sphinxuseclass}
\end{sphinxuseclass}
\end{sphinxuseclass}
\end{sphinxuseclass}
\end{sphinxuseclass}
\end{sphinxuseclass}
\end{sphinxuseclass}
\end{sphinxuseclass}
\end{sphinxuseclass}
\end{sphinxuseclass}
\end{sphinxuseclass}
\end{sphinxuseclass}
\end{sphinxuseclass}
\end{sphinxuseclass}
\end{sphinxuseclass}
\end{sphinxuseclass}
\end{sphinxuseclass}
\end{sphinxuseclass}
\end{sphinxuseclass}
\end{sphinxuseclass}
\end{sphinxuseclass}
\end{sphinxuseclass}
\end{sphinxuseclass}

\chapter{Spazio euclideo}
\label{\detokenize{ch/analytic_geometry/euclidean_space:spazio-euclideo}}\label{\detokenize{ch/analytic_geometry/euclidean_space:geometry-analytic-euclidean-space}}\label{\detokenize{ch/analytic_geometry/euclidean_space::doc}}
\sphinxAtStartPar
\sphinxstylestrong{Approccio storico\sphinxhyphen{}applicativo}
\begin{itemize}
\item {} 
\sphinxAtStartPar
\sphinxstyleemphasis{Elementi di Euclide}: formulazione assiomatica della geometria, partendo dalla definizione di concetti primitivi e postulati (5), viene sviluppata la teoria in teoremi e corollari, tramite un procedimento deduttivo.

\item {} 
\sphinxAtStartPar
Qualitativamente, la geometria di Euclide corrisponde alla concezione quotidiana dello spazio nel quale viviamo. Lo spazio euclideo fornisce il modello di spazio per la meccanica di Newton, formulata nel XVII secolo, e che rimane un ottimo modello ampiamente usato tutt’oggi per l’evoluzione di sistemi con dimensioni caratteristiche sufficientemente maggiori della scala atomica, e velocità caratteristiche sufficientemente minori della velocità della luce.

\item {} 
\sphinxAtStartPar
Una definizione più moderna di uno spazio euclideo si basa sulle traslazioni (\sphinxstylestrong{todo} citare Bowen, \sphinxstyleemphasis{Introduction to tensors and vectors}). Sia \(E\) un insieme di elementi, definiti \sphinxstylestrong{punti}, e \(V\) lo spazio vettoriale (\sphinxstylestrong{todo} riferimento al capitolo sui vettori) delle traslazioni, \(E\) viene definito uno spazio euclideo se esiste una funzione \(f: E \times E \rightarrow V\) che associa a due punti dell’insieme \(E\) uno e un solo vettore traslazione \(v \in V\) tale che
\begin{enumerate}
\sphinxsetlistlabels{\arabic}{enumi}{enumii}{}{.}%
\item {} 
\sphinxAtStartPar
\(f(x,y) = f(x,z) + f(z,y)\) per ogni \( x, y, z \in E\)

\item {} 
\sphinxAtStartPar
per \(\forall x \in E\), \(v \in V\), \(\exists ! \, y \in E\) tale che \(f(x,y) = v\)

\end{enumerate}

\item {} 
\sphinxAtStartPar
\sphinxstylestrong{todo} Dato uno spazio euclideo, si può usare un punto \(O\)  chiamato \sphinxstylestrong{origine}, per definire uno spazio vettoriale associando ogni punto \(P\) dello spazio euclideo \(E\) al vettore traslazione \(\vec{v} = P - O \in V\) \sphinxstylestrong{todo} differenza tra spazi vettoriali e spazi affini

\item {} 
\sphinxAtStartPar
Seguendo l’approccio di \sphinxstyleemphasis{Cartesio}, i punti di uno spazio possono essere rappresentati da un \sphinxstylestrong{sistema di coordinate} (\sphinxstylestrong{todo} coordinate: funzioni scalari definite nello spazio). \sphinxstyleemphasis{A volte non si riesce a rappresentare tutto lo spazio con un solo insieme di coordinate, ma servono più carte di coordinate, che si sovrappongano in alcune regioni, per poter ricavare una transizione tra due mappe differenti}. Il numero di coordinate necessario e sufficiente a rappresentare tutti i punti dello spazio coincide con la \sphinxstylestrong{dimensione} dello spazio. In questa maniera, ogni punto \(x\) in uno spazio \(n\)\sphinxhyphen{}dimensionale, \sphinxstyleemphasis{o in un suo sottoinsieme}, può essere identificato dal valore di \(n\) funzioni scalari definite nello spazio, definite coordinate.
\begin{equation*}
\begin{split}x(q^1, q^2, \dots, q^n) \ .\end{split}
\end{equation*}
\item {} 
\sphinxAtStartPar
Tra le infinite scelte possibili di un sistema di coordinate, esistono alcuni sistemi particolari, i sistemi di \sphinxstylestrong{coordinate cartesiane} \sphinxstylestrong{todo} definire le coordinate cartesiane associandole alle traslazioni \(\{ \hat{e}_k \}_{k=1:n}\)

\item {} 
\sphinxAtStartPar
Tra i sistemi di coordinate cartesiane, i sistemi di \sphinxstylestrong{coordinate cartesiane ortonormali} sono associati a traslazioni unitarie in direzioni ortogonali tra di loro. Usando un sistema di coordinate cartesiane ortogonali, è possibile definire uno spazio euclideo come uno spazio in cui sono valide le espressioni:
\begin{itemize}
\item {} 
\sphinxAtStartPar
il prodotto interno:
\begin{equation*}
\begin{split}\vec{u} \cdot \vec{v} = \sum_k u^k v^k\end{split}
\end{equation*}
\item {} 
\sphinxAtStartPar
la norma di un vettore (indotta dal prodotto interno), o della distanza tra due punti che definiscono il vettore \(\vec{v}\)
\begin{equation*}
\begin{split}|\vec{u}|^2 = \vec{u} \cdot \vec{u} = \sum_k u^k u^k \ ,\end{split}
\end{equation*}
\sphinxAtStartPar
ossia si può usare il \sphinxstylestrong{teorema di Pitagora} per il calcolo delle distanze.

\end{itemize}

\item {} 
\sphinxAtStartPar
\sphinxstylestrong{todo} cenni a spazi/geometrie non euclidee: esempi, e criteri “avanzati” per la definizione (basati su curvatura, geodesiche,…), e conseguenze,…

\end{itemize}

\sphinxstepscope

\begin{sphinxuseclass}{sd-container-fluid}
\begin{sphinxuseclass}{sd-sphinx-override}
\begin{sphinxuseclass}{sd-p-0}
\begin{sphinxuseclass}{sd-mt-2}
\begin{sphinxuseclass}{sd-mb-4}
\begin{sphinxuseclass}{sd-row}
\begin{sphinxuseclass}{sd-row-cols-2}
\begin{sphinxuseclass}{sd-gx-2}
\begin{sphinxuseclass}{sd-gy-1}
\begin{sphinxuseclass}{sd-col}
\begin{sphinxuseclass}{sd-d-flex-row}
\begin{sphinxuseclass}{sd-align-minor-center}
\begin{sphinxuseclass}{sd-container-fluid}
\begin{sphinxuseclass}{sd-sphinx-override}
\begin{sphinxuseclass}{sd-row}
\begin{sphinxuseclass}{sd-row-cols-2}
\begin{sphinxuseclass}{sd-row-cols-xs-2}
\begin{sphinxuseclass}{sd-row-cols-sm-3}
\begin{sphinxuseclass}{sd-row-cols-md-3}
\begin{sphinxuseclass}{sd-row-cols-lg-3}
\begin{sphinxuseclass}{sd-gx-3}
\begin{sphinxuseclass}{sd-gy-1}
\begin{sphinxuseclass}{sd-col}
\begin{sphinxuseclass}{sd-col-auto}
\begin{sphinxuseclass}{sd-d-flex-row}
\begin{sphinxuseclass}{sd-align-minor-center}
\sphinxAtStartPar
basics

\end{sphinxuseclass}
\end{sphinxuseclass}
\end{sphinxuseclass}
\end{sphinxuseclass}
\begin{sphinxuseclass}{sd-col}
\begin{sphinxuseclass}{sd-col-auto}
\begin{sphinxuseclass}{sd-d-flex-row}
\begin{sphinxuseclass}{sd-align-minor-center}
\sphinxAtStartPar
Oct 22, 2024

\end{sphinxuseclass}
\end{sphinxuseclass}
\end{sphinxuseclass}
\end{sphinxuseclass}
\begin{sphinxuseclass}{sd-col}
\begin{sphinxuseclass}{sd-col-auto}
\begin{sphinxuseclass}{sd-d-flex-row}
\begin{sphinxuseclass}{sd-align-minor-center}
\sphinxAtStartPar
1 min read

\end{sphinxuseclass}
\end{sphinxuseclass}
\end{sphinxuseclass}
\end{sphinxuseclass}
\end{sphinxuseclass}
\end{sphinxuseclass}
\end{sphinxuseclass}
\end{sphinxuseclass}
\end{sphinxuseclass}
\end{sphinxuseclass}
\end{sphinxuseclass}
\end{sphinxuseclass}
\end{sphinxuseclass}
\end{sphinxuseclass}
\end{sphinxuseclass}
\end{sphinxuseclass}
\end{sphinxuseclass}
\end{sphinxuseclass}
\end{sphinxuseclass}
\end{sphinxuseclass}
\end{sphinxuseclass}
\end{sphinxuseclass}
\end{sphinxuseclass}
\end{sphinxuseclass}
\end{sphinxuseclass}
\end{sphinxuseclass}

\chapter{Geometria analitica nel piano}
\label{\detokenize{ch/analytic_geometry/analytic_geometry_2d:geometria-analitica-nel-piano}}\label{\detokenize{ch/analytic_geometry/analytic_geometry_2d:geometry-analytic-2d}}\label{\detokenize{ch/analytic_geometry/analytic_geometry_2d::doc}}
\sphinxAtStartPar
La geometria analitica nel piano si occupa della descrizione dello spazio bidimensionale euclideo e delle entità geometriche in esso, grazie all’uso di sistemi di coordinate \((q^1, q^2)\).
\begin{itemize}
\item {} 
\sphinxAtStartPar
\sphinxstylestrong{Sistemi di coordinate, e punti nello spazio.} Vengono presentati alcuni sistemi di coordinate che risulteranno utili nello studio della geometria analitica nel piano, e le regole di trasformazione tra di esse.

\item {} 
\sphinxAtStartPar
\sphinxstylestrong{Angoli e distanze.} Viene definita la struttura di uno spazio euclideo tramite la definizione degli angoli e delle distanze usando sistemi di coordinate cartesiane ortonormali, e le definizioni di prodotto interno (\sphinxstylestrong{todo} e prodotto vettoriale?).

\item {} 
\sphinxAtStartPar
\sphinxstylestrong{Curve.} Vengono definite le curve nel piano, come relazioni tra le coordinate di un sistema di coordinate. \sphinxstylestrong{todo} \sphinxstyleemphasis{l’equazione di una curva rappresenta un tra la geometria e l’algebra tipico della geometria analitica.} Vengono poi studiate alcune curve particolari:
\begin{itemize}
\item {} 
\sphinxAtStartPar
\sphinxstylestrong{Rette}:
\begin{itemize}
\item {} 
\sphinxAtStartPar
equazione

\item {} 
\sphinxAtStartPar
posizione relativa punto\sphinxhyphen{}retta, distanza punto\sphinxhyphen{}retta, posizione relativa retta\sphinxhyphen{}retta

\end{itemize}

\item {} 
\sphinxAtStartPar
\sphinxstylestrong{Coniche}:
\begin{itemize}
\item {} 
\sphinxAtStartPar
introduzione: …motivazione della loro importanza (gravitazione, ottica,…)

\item {} 
\sphinxAtStartPar
def, equazioni e caratteristiche con un’oppportuna scelta del sistema di coordinate; successivamente traslazione e rotazione

\item {} 
\sphinxAtStartPar
…

\end{itemize}

\end{itemize}

\end{itemize}

\sphinxstepscope


\section{Sistemi di coordinate}
\label{\detokenize{ch/analytic_geometry/analytic_geometry_2d/coordinates:sistemi-di-coordinate}}\label{\detokenize{ch/analytic_geometry/analytic_geometry_2d/coordinates:geometry-analytic-2d-coordinates}}\label{\detokenize{ch/analytic_geometry/analytic_geometry_2d/coordinates::doc}}

\subsection{Esempi}
\label{\detokenize{ch/analytic_geometry/analytic_geometry_2d/coordinates:esempi}}
\sphinxAtStartPar
\sphinxstylestrong{Sistema di coordinate cartesiane ortonormale.} \((x, y)\)

\sphinxAtStartPar
\sphinxstylestrong{Sistema di coordinate polari.} \((r, \theta)\). La legge di trasformazione delle coordinate tra un sistema di coordinate cartesiane ortonormale e un sistema di coordinate polari, con la stessa origine e l’asse \(x\) come direzione di riferimento per la misura dell’angolo \(\theta\) è
\begin{equation*}
\begin{split}\begin{cases}
  x = r \cos \theta \\
  y = r \sin \theta \ .
\end{cases}\end{split}
\end{equation*}
\sphinxAtStartPar
\sphinxstylestrong{todo.} Aggiungere immagine


\subsection{Trasformazione di coordinate}
\label{\detokenize{ch/analytic_geometry/analytic_geometry_2d/coordinates:trasformazione-di-coordinate}}
\sphinxAtStartPar
Vengono discusse alcune leggi di trasformazione tra le coordinate di diversi sistemi di coordinate.

\sphinxAtStartPar
\sphinxstylestrong{Traslazione dell’origine di due sistemi cartesiani con assi allineati.}
\begin{equation*}
\begin{split}\begin{cases}
  x' = x - x_{O'} \\
  y' = x - y_{O'}
\end{cases}\end{split}
\end{equation*}\begin{equation*}
\begin{split}\underline{x}' = \underline{x} - \underline{x}_{O'}\end{split}
\end{equation*}
\sphinxAtStartPar
\sphinxstylestrong{Rotazione degli assi di due sistemi cartesiani con stessa origine.}
\begin{equation*}
\begin{split}\begin{cases}
  x' = x \cos \theta + y \sin \theta \\
  y' =-x \sin \theta + y \cos \theta
\end{cases}\end{split}
\end{equation*}\begin{equation*}
\begin{split}\underline{x}' = R \underline{x}\end{split}
\end{equation*}
\sphinxAtStartPar
\sphinxstylestrong{Traslazione dell’origine e rotazione degli assi di due sistemi di coordinate cartesiane.}

\sphinxAtStartPar
\sphinxstylestrong{todo} \sphinxstyleemphasis{L’ordine delle trasformazioni è importante}
\begin{equation*}
\begin{split} x \rightarrow T \rightarrow x' \rightarrow R \rightarrow x''\end{split}
\end{equation*}\begin{equation*}
\begin{split}\begin{aligned}
\underline{x}' & = \underline{x} - \underline{x}_{O'} \\
\underline{x}'' & = R \underline{x}' = R \, (\underline{x} - \underline{x}_{O'}) \\
\end{aligned}\end{split}
\end{equation*}\begin{equation*}
\begin{split} x \rightarrow R \rightarrow x' \rightarrow T \rightarrow x''\end{split}
\end{equation*}\begin{equation*}
\begin{split}\begin{aligned}
\underline{x}'  & = R \, \underline{x}  \\
\underline{x}'' & = \underline{x}' - \underline{x}'_{O''} = R \, \underline{x} - \underline{x}'_{O''} \\ 
\end{aligned}\end{split}
\end{equation*}
\sphinxAtStartPar
\sphinxstylestrong{Altri esempi di coordinate e trasformazioni di coordinate.} \sphinxstylestrong{todo.} come esercizio?

\sphinxstepscope


\section{Distanze e angoli}
\label{\detokenize{ch/analytic_geometry/analytic_geometry_2d/points:distanze-e-angoli}}\label{\detokenize{ch/analytic_geometry/analytic_geometry_2d/points:geometry-analytic-2d-points}}\label{\detokenize{ch/analytic_geometry/analytic_geometry_2d/points::doc}}

\subsection{Distanza tra punti}
\label{\detokenize{ch/analytic_geometry/analytic_geometry_2d/points:distanza-tra-punti}}
\sphinxAtStartPar
Usando un sistema di coordinate cartesiane, la distanza tra due punti nel piano viene calcolata usando il teorema di Pitagora ,
\begin{equation*}
\begin{split}d_{12} = |P_2 - P_1| = \sqrt{(x_2 - x_1)^2 + (y_2 - y_1)^2}\end{split}
\end{equation*}
\sphinxAtStartPar
In maniera equivalente viene il modulo (o lunghezza) di un vettore \(\vec{v} = v_x \hat{x} + v_y \hat{y}\),
\begin{equation*}
\begin{split}|\vec{v}| = \sqrt{v_x^2 + v_y^2} \ .\end{split}
\end{equation*}

\subsection{Angoli tra direzioni}
\label{\detokenize{ch/analytic_geometry/analytic_geometry_2d/points:angoli-tra-direzioni}}
\sphinxAtStartPar
Nello spazio euclideo, una direzione con verso può essere identificata da un vettore \(\vec{v}\). Date due direzioni con verso identificate dai vettori \(\vec{u}\), \(\vec{v}\), l’angolo formato tra i due vettori può essere identificato dalla proiezione di un vettore sull’altro tramite il prodotto interno
\begin{equation*}
\begin{split}\vec{u} \cdot \vec{v} := |\vec{u}| |\vec{v}| \cos \theta_{uv} \ ,\end{split}
\end{equation*}
\sphinxAtStartPar
e usando un sistema di coordinate cartesiano
\begin{equation*}
\begin{split}\vec{u} \cdot \vec{v} = u_x v_x + u_y v_y \ .\end{split}
\end{equation*}
\sphinxAtStartPar
\sphinxstylestrong{todo} Dimostrazione con \(\vec{u} = u_x \hat{x} + u_y \hat{y}\), \(\vec{v} = v_x \hat{x} + v_y \hat{y}\), \(\hat{x} \cdot \hat{x} = \hat{y} \cdot \hat{y} = 1\), \(\hat{x} \cdot \hat{y} = 0\).

\sphinxstepscope


\section{Curve nel piano}
\label{\detokenize{ch/analytic_geometry/analytic_geometry_2d/curves:curve-nel-piano}}\label{\detokenize{ch/analytic_geometry/analytic_geometry_2d/curves:geometry-analytic-2d-curves}}\label{\detokenize{ch/analytic_geometry/analytic_geometry_2d/curves::doc}}
\sphinxAtStartPar
Una curva nello spazio euclideo \(E^2\), nel piano, è un luogo dei punti del piano che possono essere identificati da una relazione tra le coordinate di un sistema di coordinate.

\sphinxAtStartPar
\sphinxstylestrong{Esempi.} \sphinxstylestrong{todo.} grafici




\subsection{Rappresentazioni di una curva}
\label{\detokenize{ch/analytic_geometry/analytic_geometry_2d/curves:rappresentazioni-di-una-curva}}
\sphinxAtStartPar
Dato un sistema di coordinate \((q^1, q^2)\) che descrive il piano, una curva \(\gamma\) può essere rappresentata in diverse maniere:

\sphinxAtStartPar
\sphinxstylestrong{Rappresentazione esplicita.}
\begin{equation*}
\begin{split}\gamma: \ q^2 = f(q^1)\end{split}
\end{equation*}
\sphinxAtStartPar
\sphinxstylestrong{todo.} limiti di questa rappresentazione

\sphinxAtStartPar
\sphinxstylestrong{Rappresentazione implicita.}
\begin{equation*}
\begin{split}\gamma: \ F(q^1, q^2) = 0\end{split}
\end{equation*}
\sphinxAtStartPar
\sphinxstylestrong{todo.} limiti di questa rappresentazione

\sphinxAtStartPar
\sphinxstylestrong{Rappresentazione parametrica.}
\begin{equation*}
\begin{split}\gamma(s): \ \begin{cases} q^1 = f^1(s) \\ q^2 = f^2(s) \end{cases}\end{split}
\end{equation*}

\subsection{Esempi}
\label{\detokenize{ch/analytic_geometry/analytic_geometry_2d/curves:esempi}}
\sphinxstepscope


\section{Rette nel piano}
\label{\detokenize{ch/analytic_geometry/analytic_geometry_2d/lines:rette-nel-piano}}\label{\detokenize{ch/analytic_geometry/analytic_geometry_2d/lines:geometry-analytic-2d-lines}}\label{\detokenize{ch/analytic_geometry/analytic_geometry_2d/lines::doc}}

\subsection{Definizioni ed equazione}
\label{\detokenize{ch/analytic_geometry/analytic_geometry_2d/lines:definizioni-ed-equazione}}
\sphinxAtStartPar
Per Euclide, il concetto di retta è un concetto primitivo.



\sphinxAtStartPar
Per trovare l’equazione di una retta, si possono usare delle definizione equivalenti.

\sphinxAtStartPar
\sphinxstylestrong{Def. 1 Passaggio per un punto e direzione: equazione parametrica.}
\begin{equation*}
\begin{split}P = P_0 + \lambda \overrightarrow{v}\end{split}
\end{equation*}\begin{equation*}
\begin{split}r: \begin{cases}
  x(p) = x_0 + p \ d_x \\
  y(p) = y_0 + p \ d_y \\
\end{cases}
\qquad \text{o} \qquad
  \mathbf{r}(p) = \mathbf{r}_0 + p \mathbf{d} 
\end{split}
\end{equation*}
\sphinxAtStartPar
\sphinxstylestrong{Def. 2} Una retta è il luogo geometrico dei punti equidistanti da due punti distinti nel piano, \(P_1\), \(P_2\). Usando un sistema di coordinate cartesiano, \(P_1 = (x_1, y_1)\), \(P_2 = (x_2, y_2)\), mentre le coordinate di un punto \(P\) appartenente alla retta sono \((x,y)\)
\begin{equation*}
\begin{split}|P - P_1| = |P-P_2|\end{split}
\end{equation*}\begin{equation*}
\begin{split}\begin{aligned}
 (x - x_1)^2 + (y - y_1)^2 &=  (x - x_2)^2 + (y - y_2)^2 \\
 x^2 - 2 x x_1 + x_1^2 + y^2 - 2 y y_1 + y_1^2 &=  x^2 - 2 x x_2 + x_2^2 + y^2 - 2 y y_2 + y_2^2 \\
\end{aligned}\end{split}
\end{equation*}\begin{equation*}
\begin{split}  2 ( x_2 - x_1 ) x + 2 ( y_2 - y_1 ) y - x_1^2 - y_1^2 - x_2^2 - y_2^2 = 0 \end{split}
\end{equation*}

\subsection{Posizioni reciproche}
\label{\detokenize{ch/analytic_geometry/analytic_geometry_2d/lines:posizioni-reciproche}}

\subsubsection{Posizione reciproca di punto e retta}
\label{\detokenize{ch/analytic_geometry/analytic_geometry_2d/lines:posizione-reciproca-di-punto-e-retta}}
\sphinxAtStartPar
\sphinxstylestrong{Punto appartenente alla retta.} Una retta \(r\) passa per un punto \(P\) assegnato se le coordinate del punto \(P\) soddisfano le equazioni che descrivono la retta.

\sphinxAtStartPar
\sphinxstylestrong{Distanza punto\sphinxhyphen{}retta.} per calcolare la distanza punto\sphinxhyphen{}retta si può


\subsubsection{Posizione reciproca di rette}
\label{\detokenize{ch/analytic_geometry/analytic_geometry_2d/lines:posizione-reciproca-di-rette}}
\sphinxAtStartPar
\sphinxstylestrong{Rette coincidenti.} Due rette sono coincidenti se si possono scrivere le loro equazioni come
\begin{equation*}
\begin{split}\begin{aligned}
  r_1: \ \mathbf{r}_1(p_1) = \mathbf{r}_{1,0} + p_1 \ \mathbf{d}_1 \\
  r_2: \ \mathbf{r}_2(p_2) = \mathbf{r}_{2,0} + p_2 \ \mathbf{d}_2
\end{aligned}\end{split}
\end{equation*}
\sphinxAtStartPar
con
\begin{equation*}
\begin{split}\mathbf{r}_{1,0} = \mathbf{r}_{2,0} \qquad , \qquad \mathbf{d}_1 \propto \mathbf{d}_2\end{split}
\end{equation*}
\sphinxAtStartPar
\sphinxstylestrong{Rette parallele nel piano.}
\begin{equation*}
\begin{split}\mathbf{d}_1 \propto \mathbf{d}_2\end{split}
\end{equation*}
\sphinxAtStartPar
\sphinxstylestrong{Rette incidenti.}
\begin{equation*}
\begin{split}\mathbf{d}_1 \cdot \mathbf{d}_2 \ne 0\end{split}
\end{equation*}
\sphinxAtStartPar
\sphinxstylestrong{Rette incidenti perpendicolari.}
\begin{equation*}
\begin{split}(\mathbf{r}_1 - \mathbf{r}_{0,1}) \cdot \mathbf{r}_2 - \mathbf{r}_{0,2}) = 0 \qquad , \qquad \mathbf{d}_1 \cdot \mathbf{d}_2 = 0\end{split}
\end{equation*}

\subsection{Distanza punto retta}
\label{\detokenize{ch/analytic_geometry/analytic_geometry_2d/lines:distanza-punto-retta}}
\sphinxAtStartPar
Dato il punto \(P_1 = (x_1, y_1)\), \(r: a x + b y + c = 0\),
\begin{equation*}
\begin{split}|r - P_1| = \min_{P \in r} |P - P_1|\end{split}
\end{equation*}\begin{equation*}
\begin{split}\begin{aligned}
|r - P_1|^2 = ( x - x_1 )^2 + ( y - y_1 )^2 
\end{aligned}\end{split}
\end{equation*}
\sphinxstepscope


\section{Coniche}
\label{\detokenize{ch/analytic_geometry/analytic_geometry_2d/conics:coniche}}\label{\detokenize{ch/analytic_geometry/analytic_geometry_2d/conics:geometry-analytic-2d-conics}}\label{\detokenize{ch/analytic_geometry/analytic_geometry_2d/conics::doc}}
\sphinxAtStartPar
Le coniche sono curve che possono essere ottenute come intersezione tra un piano e un (doppio) cono circolare retto \sphinxstylestrong{todo} Aggiungere collegamento alla pagina della geometria 3D, con dimostrazione.

\sphinxAtStartPar
Queste curve compaiono in alcuni ambiti di interesse pratico:
\begin{itemize}
\item {} 
\sphinxAtStartPar
ottica e acustica geometrica

\item {} 
\sphinxAtStartPar
gravitazione: secondo la meccanica di Newton, i corpi celesti descrivono traiettorie nello spazio che hanno la forma delle curve coniche:

\item {} 
\sphinxAtStartPar
in altri ambiti della scienza in cui compaiono funzioni quadratiche

\end{itemize}

\sphinxAtStartPar
Per la loro relativa semplicità e frequenza con la quale appaiono in diverse applicazioni, lo studio delle coniche si presenta come utile argomento per l’applicazione delle nozioni di geometria analitica apprese finora.

\sphinxAtStartPar
Le coniche possono essere definite tramite pochi elementi geometrici caratteristici, come un punto di riferimento \(F\) detto \sphinxstylestrong{fuoco} e una retta di riferimento \(d\) detta \sphinxstylestrong{direttrice}.
\begin{itemize}
\item {} 
\sphinxAtStartPar
equazione delle coniche usando le coordinate cartesiane

\item {} 
\sphinxAtStartPar
equazione delle coniche usando le coordinate polari

\item {} 
\sphinxAtStartPar
proprietà geometriche delle coniche

\item {} 
\sphinxAtStartPar
coniche come sezione di un cono circolare

\item {} 
\sphinxAtStartPar
coniche e gravitazione di Newton

\end{itemize}

\sphinxstepscope


\section{Coniche nel piano}
\label{\detokenize{ch/analytic_geometry/analytic_geometry_2d/conics-cartesian:coniche-nel-piano}}\label{\detokenize{ch/analytic_geometry/analytic_geometry_2d/conics-cartesian:geometry-analytic-2d-conics-cartesian}}\label{\detokenize{ch/analytic_geometry/analytic_geometry_2d/conics-cartesian::doc}}\begin{itemize}
\item {} 
\sphinxAtStartPar
Sezioni di un cono

\item {} 
\sphinxAtStartPar
Definibili tramite le distanze dei loro punti da punti (centro o fuochi) e rette caratteristiche (direttrici)

\item {} 
\sphinxAtStartPar
Espressioni usando coordinate cartesiane e polari

\item {} 
\sphinxAtStartPar
Proprietà “ottiche”

\item {} 
\sphinxAtStartPar
Applicazioni nella teoria della gravitazione di Newton

\end{itemize}


\subsection{Circonferenza}
\label{\detokenize{ch/analytic_geometry/analytic_geometry_2d/conics-cartesian:circonferenza}}
\sphinxAtStartPar
\sphinxstylestrong{Definizione.} Una circonferenza è il luogo dei punti equidistanti da un punto \(C\) dato, detto centro della criconferenza. La distanza tra i punti del circonferenza e il centro viene definito raggio della circonferenza.
\begin{equation*}
\begin{split}|P - C| = R\end{split}
\end{equation*}
\sphinxAtStartPar
\sphinxstylestrong{Equazione in coordinate cartesiane.} Per ricavare l’equazione di una circonferenza in coordinate cartesiane, si usa la formula per il calcolo della distanza tra punti. Se si sceglie un sistema di coordinate cartesiane con origine in \(C\) s.t. \((x_C, y_C) = (0,0)\), la condizione che identifica le coordinate cartesiane \((x,y)\) dei punti di una circonferenza di raggio \(R\) centrata in \(C\)
\begin{equation*}
\begin{split}R^2 = |P-C|^2 = x^2 + y^2 \ ,\end{split}
\end{equation*}
\sphinxAtStartPar
cioè
\begin{equation*}
\begin{split}x^2 + y^2 = R^2 \ .\end{split}
\end{equation*}
\sphinxAtStartPar
\sphinxstylestrong{Equazione in coordinate polari.} Usando un sistema di coordinate polari \(\{r, \theta\}\) con origine nel centro della circonferenza, la condizione che identifica una criconferenza di raggio \(R\) è
\begin{equation*}
\begin{split}r = R\end{split}
\end{equation*}

\subsection{Parabola}
\label{\detokenize{ch/analytic_geometry/analytic_geometry_2d/conics-cartesian:parabola}}\begin{equation*}
\begin{split}|P - d| = |P - F|\end{split}
\end{equation*}
\sphinxAtStartPar
\sphinxstylestrong{Equazione in coordinate cartesiane.}
\begin{equation*}
\begin{split}\left(y+\dfrac{d}{2}\right)^2 = x^2 + \left(y-\dfrac{d}{2}\right)^2\end{split}
\end{equation*}\begin{equation*}
\begin{split}y^2 + d y + \dfrac{d^2}{4} = x^2 + y^2 - d y + \dfrac{d^2}{4}\end{split}
\end{equation*}\begin{equation*}
\begin{split}y = \frac{1}{4d} x^2\end{split}
\end{equation*}

\subsection{Ellisse}
\label{\detokenize{ch/analytic_geometry/analytic_geometry_2d/conics-cartesian:ellisse}}\begin{equation*}
\begin{split}|P - F_1| + |P - F_2| = 2a\end{split}
\end{equation*}
\sphinxAtStartPar
Con la scelta \(F_1 \equiv (-c,0)\), \(F_2 \equiv (c,0)\)
\begin{equation*}
\begin{split}\sqrt{(x+c)^2 + y^2} = 2a - \sqrt{(x-c)^2 + y^2}\end{split}
\end{equation*}\begin{equation*}
\begin{split}x^2 + 2 c x + c^2 + y^2 = 4 a^2 - 4a \sqrt{(x-c)^2 + y^2} + x^2 - 2 c x + c^2 + y^2\end{split}
\end{equation*}\begin{equation*}
\begin{split}4a \sqrt{(x-c)^2 + y^2} = 4 a^2 - 4 c x\end{split}
\end{equation*}\begin{equation*}
\begin{split}a^2 x^2-2 a^2 c x + a^2 c^2 + a^2 y^2 = a^4 - 2 a^2 c x + c^2 x^2\end{split}
\end{equation*}\begin{equation*}
\begin{split}(a^2 - c^2)x^2 + a^2 y^2 = a^2 ( a^2 - c^2 )\end{split}
\end{equation*}\begin{equation*}
\begin{split}\dfrac{x^2}{a^2} + \dfrac{y^2}{b^2} = 1\end{split}
\end{equation*}
\sphinxAtStartPar
avendo definito \(b^2 := a^2 - c^2\).


\subsection{Iperbole}
\label{\detokenize{ch/analytic_geometry/analytic_geometry_2d/conics-cartesian:iperbole}}\begin{equation*}
\begin{split}\big| |P - F_1| - |P - F_2| \big| = 2a\end{split}
\end{equation*}
\sphinxAtStartPar
Con la scelta \(F_1 \equiv (-c,0)\), \(F_2 \equiv (c,0)\)
\begin{equation*}
\begin{split}\sqrt{(x+c)^2 + y^2} = \mp 2a + \sqrt{(x-c)^2 + y^2}\end{split}
\end{equation*}\begin{equation*}
\begin{split}x^2 + 2 c x + c^2 + y^2 = 4 a^2 \mp 4a \sqrt{(x-c)^2 + y^2} + x^2 - 2 c x + c^2 + y^2\end{split}
\end{equation*}\begin{equation*}
\begin{split}\pm 4a \sqrt{(x-c)^2 + y^2} = 4 a^2 - 4 c x\end{split}
\end{equation*}\begin{equation*}
\begin{split}a^2 x^2-2 a^2 c x + a^2 c^2 + a^2 y^2 = a^4 - 2 a^2 c x + c^2 x^2\end{split}
\end{equation*}\begin{equation*}
\begin{split}(a^2 - c^2)x^2 + a^2 y^2 = a^2 ( a^2 - c^2 )\end{split}
\end{equation*}\begin{equation*}
\begin{split}\dfrac{x^2}{a^2} - \dfrac{y^2}{b^2} = 1\end{split}
\end{equation*}
\sphinxAtStartPar
avendo definito \(b^2 := c^2 - a^2\).

\sphinxstepscope


\section{Coniche nel piano: coordinate polari}
\label{\detokenize{ch/analytic_geometry/analytic_geometry_2d/conics-polar:coniche-nel-piano-coordinate-polari}}\label{\detokenize{ch/analytic_geometry/analytic_geometry_2d/conics-polar:geometry-analytic-2d-conics-polar}}\label{\detokenize{ch/analytic_geometry/analytic_geometry_2d/conics-polar::doc}}
\sphinxAtStartPar
Le coniche possono essere anche caratterizzate dal valore dell’eccentricità,
\begin{equation*}
\begin{split}e = \dfrac{\text{dist(punto, fuoco)}}{\text{dist(punto, direttrice}} = \dfrac{\text{dist}(P,F)}{\text{dst}(P,d)} \ .\end{split}
\end{equation*}
\sphinxAtStartPar
Questa definizione permette di ricavare facilmente l’equazione delle coniche usando un sistema di coordinate polari, centrato nel fuoco \(F\).

\sphinxAtStartPar
Poiché
\begin{equation*}
\begin{split}\text{dist}(P,F) = r\end{split}
\end{equation*}\begin{equation*}
\begin{split}\text{dist}(P,d) = D - r \cos \theta\end{split}
\end{equation*}
\sphinxAtStartPar
l’equazione generale delle coniche diventa
\begin{equation*}
\begin{split}e \left( D - r \cos \theta \right) = r\end{split}
\end{equation*}
\sphinxAtStartPar
\sphinxstylestrong{Circonferenza, \(e = 0\).} L’equazione generale per la circonferenza degenera in
\begin{equation*}
\begin{split}r = R \ ,\end{split}
\end{equation*}
\sphinxAtStartPar
per \(e \rightarrow 0\), \(D \rightarrow +\infty\) in modo tale che \(e D := R\) finito.

\sphinxAtStartPar
\sphinxstylestrong{Ellisse, \(e \in (0,1)\).}

\sphinxAtStartPar
\sphinxstylestrong{Parabola, \(e = 1\).}

\sphinxAtStartPar
\sphinxstylestrong{Iperbole, \(e > 1\).}

\sphinxstepscope


\chapter{Geometria analitica nello spazio}
\label{\detokenize{ch/analytic_geometry/analytic_geometry_3d:geometria-analitica-nello-spazio}}\label{\detokenize{ch/analytic_geometry/analytic_geometry_3d:geometry-analytic-3d}}\label{\detokenize{ch/analytic_geometry/analytic_geometry_3d::doc}}
\sphinxstepscope


\section{Sistemi di coordinate per lo spazio euclideo \protect\(E^3\protect\)}
\label{\detokenize{ch/analytic_geometry/analytic_geometry_3d/points:sistemi-di-coordinate-per-lo-spazio-euclideo-e-3}}\label{\detokenize{ch/analytic_geometry/analytic_geometry_3d/points:geometry-analytic-3d-points}}\label{\detokenize{ch/analytic_geometry/analytic_geometry_3d/points::doc}}

\subsection{Coordinate cartesiane}
\label{\detokenize{ch/analytic_geometry/analytic_geometry_3d/points:coordinate-cartesiane}}
\sphinxAtStartPar
Le coordinate cartesiane \((x,y,z)\) di un punto \(P\) dello spazio euclideo \(E^3\) permettono di definire il vettore euclideo tra l’origine \(O \equiv (0,0,0)\) e il punto \(P\)
\begin{equation*}
\begin{split}(P-O) = x \hat{\mathbf{x}} + y \hat{\mathbf{y}} + z \hat{\mathbf{z}} \ ,\end{split}
\end{equation*}
\sphinxAtStartPar
usando i vettori della base cartesiana \(\{\hat{\mathbf{x}}, \hat{\mathbf{y}}, \hat{\mathbf{z}} \}\)


\subsubsection{Distanza punto\sphinxhyphen{}punto}
\label{\detokenize{ch/analytic_geometry/analytic_geometry_3d/points:distanza-punto-punto}}
\sphinxAtStartPar
Usando le coordinate cartesiane, la distanza tra due punti \(P \equiv(x_P, y_P, z_P)\), \(Q \equiv (x_Q, y_Q, z_Q)\) si può calcolare usando il teorema di Pitagora come
\begin{equation*}
\begin{split}|P-Q|^2 = (x_P - x_Q)^2 + (y_P - y_Q)^2 + (z_P - z_Q)^2 \ .\end{split}
\end{equation*}

\subsection{Coordinate cilindriche}
\label{\detokenize{ch/analytic_geometry/analytic_geometry_3d/points:coordinate-cilindriche}}
\sphinxAtStartPar
Dato un sistema di coordinate cartesiane, si può definire un sistema di coordinate cilindriche \((R, \theta, z)\) tramite la legge di trasformazione delle coordinate
\begin{equation*}
\begin{split}\begin{cases}
x = R \cos \theta \\
y = R \sin \theta \\
z = z
\end{cases}\end{split}
\end{equation*}

\subsection{Coordinate sferiche}
\label{\detokenize{ch/analytic_geometry/analytic_geometry_3d/points:coordinate-sferiche}}
\sphinxAtStartPar
Dato un sistema di coordinate cartesiane, si può definire un sistema di coordinate sferiche \((r, \theta, \phi)\) tramite la legge di trasformazione delle coordinate
\begin{equation*}
\begin{split}\begin{cases}
x = r \sin \phi + \cos \theta \\
y = r \sin \phi + \sin \theta \\
z = r \cos \phi
\end{cases}\end{split}
\end{equation*}
\sphinxstepscope


\section{Piani nello spazio}
\label{\detokenize{ch/analytic_geometry/analytic_geometry_3d/planes:piani-nello-spazio}}\label{\detokenize{ch/analytic_geometry/analytic_geometry_3d/planes:geometry-analytic-3d-planes}}\label{\detokenize{ch/analytic_geometry/analytic_geometry_3d/planes::doc}}
\sphinxAtStartPar
Un piano \(\pi\) può essere definito come il luogo dei punti \(P\) dello spazio che formano un vettore \((P-Q)\) con un punto dato \(Q\) ortogonali a un vettore \(\overrightarrow{n}\) che indica la direzione normale al piano \(\pi\). Usando le proprietà del prodotto scalare,
\begin{equation*}
\begin{split}(P-Q) \cdot \overrightarrow{n} = 0 \ .\end{split}
\end{equation*}
\sphinxAtStartPar
Usando un sistema di coordinate cartesiane, si può trovare l’equazione implicita del piano \(\pi\),
\begin{equation*}
\begin{split}\pi: \ (x - x_Q) n_x + (y - y_Q) n_y + (z - z_Q) n_z = 0 \ .\end{split}
\end{equation*}
\sphinxAtStartPar
\sphinxstylestrong{Osservazione.} L’equazione implicita del piano è independente dal modulo del vettore \(\vec{n}\), poiché rappresenterebbe un ininfluente fattore moltiplicativo (diverso da zero) nel termine di sinistra quando uguagliato a zero.

\sphinxAtStartPar
Partendo dalla prima definizione, si possono ricavare le equazioni parametriche del piano. Dato il vettore \(\vec{n}\), si possono trovare due vettori \(\vec{t}_1\), \(\vec{t}_2\) a esso ortogonali,
\begin{equation*}
\begin{split}\vec{t} \cdot \vec{n} = 0 \ .\end{split}
\end{equation*}
\sphinxAtStartPar
Se i due vettori non sono tra di loro allineati, o meglio proporzionali, è possibile descrivere tutti i punti del piano come una loro combinazione lineare
\begin{equation*}
\begin{split}\pi: \ P = Q + \lambda_1 \vec{t}_1 + \lambda_2 \vec{t}_2 \ .\end{split}
\end{equation*}

\subsection{Distanza punto\sphinxhyphen{}piano}
\label{\detokenize{ch/analytic_geometry/analytic_geometry_3d/planes:distanza-punto-piano}}
\sphinxAtStartPar
Dato un punto \(A\) e un piano \(\pi\), di cui sono noti un punto \(Q\) e il vettore normale \(\vec{n}\), la distanza di \(A\) da \(\pi\) può essere calcolata come il valore assoluto della proiezione del vettore \(A-Q\) lungo la direzione normale al piano, individuata da \(\vec{n}\),
\begin{equation*}
\begin{split}\text{dist}(A,\pi) = \left| \hat{n} \cdot (A-Q) \right| \ ,\end{split}
\end{equation*}
\sphinxAtStartPar
avendo usato il vettore unitario \(\hat{n} = \frac{\vec{n}}{|\vec{n}|}\) per la proiezione.

\sphinxstepscope


\section{Curve nello spazio}
\label{\detokenize{ch/analytic_geometry/analytic_geometry_3d/curves:curve-nello-spazio}}\label{\detokenize{ch/analytic_geometry/analytic_geometry_3d/curves:geometry-analytic-3d-curves}}\label{\detokenize{ch/analytic_geometry/analytic_geometry_3d/curves::doc}}
\sphinxAtStartPar
Dato un sistema di coordinate \((q^1, q^2, q^3)\) curva \(\gamma\) nello spazio può essere descritta in \sphinxstylestrong{forma parametrica}, fornendo l’espressione delle coordinate in funzione di un parametro \(\lambda\),
\begin{equation*}
\begin{split}q^k(\lambda) \ .\end{split}
\end{equation*}
\sphinxAtStartPar
Usando le coordinate cartesiane, i punti della curva sono identificati dalla famiglia di vettori euclidei
\begin{equation*}
\begin{split}\gamma: \ \vec{r}(\lambda) = x(\lambda) \hat{x} + y(\lambda) \hat{y} + z(\lambda) \hat{z} \ ,\end{split}
\end{equation*}
\sphinxAtStartPar
al variare del parametro \(\lambda\).

\sphinxstepscope


\section{Rette nello spazio}
\label{\detokenize{ch/analytic_geometry/analytic_geometry_3d/lines:rette-nello-spazio}}\label{\detokenize{ch/analytic_geometry/analytic_geometry_3d/lines:geometry-analytic-3d-lines}}\label{\detokenize{ch/analytic_geometry/analytic_geometry_3d/lines::doc}}

\subsection{Equazione della retta}
\label{\detokenize{ch/analytic_geometry/analytic_geometry_3d/lines:equazione-della-retta}}
\sphinxAtStartPar
\sphinxstylestrong{Forma parametrica}
\begin{equation*}
\begin{split}r: P = Q + \lambda \, \vec{v} \ .\end{split}
\end{equation*}
\sphinxAtStartPar
\sphinxstylestrong{…} Intersezione tra due piani


\subsection{Distanza punto\sphinxhyphen{}retta}
\label{\detokenize{ch/analytic_geometry/analytic_geometry_3d/lines:distanza-punto-retta}}
\sphinxAtStartPar
Dato un punto \(A\) e una retta \(r\), di cui sono noti un punto \(Q\) e il vettore \(\vec{v}\), la distanza di \(A\) da \(r\) può essere calcolata come il valore assoluto della proiezione del vettore \(A-Q\) in direzione ortogonale alla direzione della retta, individuata da \(\vec{v}\),
\begin{equation*}
\begin{split}\begin{aligned}
\text{dist}(A,r) & = \left| (A-Q) - \hat{v} \ \hat{v} \cdot (A-Q) \right| = \\ 
                 & = \left| \hat{v} \times (A-Q) \right| 
\end{aligned}\end{split}
\end{equation*}
\sphinxAtStartPar
avendo usato il vettore unitario \(\hat{v} = \frac{\vec{v}}{|\vec{v}|}\) per la proiezione.

\sphinxstepscope


\section{Cono circolare retto e coniche}
\label{\detokenize{ch/analytic_geometry/analytic_geometry_3d/cone:cono-circolare-retto-e-coniche}}\label{\detokenize{ch/analytic_geometry/analytic_geometry_3d/cone:geometry-analytic-3d-cone}}\label{\detokenize{ch/analytic_geometry/analytic_geometry_3d/cone::doc}}

\subsection{Equazione del cono}
\label{\detokenize{ch/analytic_geometry/analytic_geometry_3d/cone:equazione-del-cono}}
\sphinxAtStartPar
Equazioni del (doppio) cono circolare retto, usando un sistema di coordinate cilindriche,
\begin{equation*}
\begin{split}r = a \, z \ ,\end{split}
\end{equation*}
\sphinxAtStartPar
per \(z \in (-\infty, +\infty)\), \(\theta \in (0, 2 \pi)\).


\subsection{Coniche: intersezione tra cono e piano}
\label{\detokenize{ch/analytic_geometry/analytic_geometry_3d/cone:coniche-intersezione-tra-cono-e-piano}}
\sphinxstepscope


\part{Calcolo infinitesimale}

\sphinxstepscope


\chapter{Calcolo infinitesimale}
\label{\detokenize{ch/infinitesimal_calculus:calcolo-infinitesimale}}\label{\detokenize{ch/infinitesimal_calculus:infinitesimal-calculus}}\label{\detokenize{ch/infinitesimal_calculus::doc}}
\sphinxAtStartPar
Il calcolo infinitesimale si occupa dello studio \sphinxstylestrong{di… todo}

\sphinxAtStartPar
Questo capitolo presenta il calcolo infinitesimale per le funzioni reali di una variabile reale, \(f: D \in \mathbb{R} \rightarrow \mathbb{R}\), come inizialmente formulate da Newton \sphinxstylestrong{{[}REF{]}} e Lebniz \sphinxstylestrong{{[}REF{]}} nel XVII secolo nell’ambito dello sviluppo della meccanica classica \sphinxstylestrong{{[}REF{]}} e formalizzate nel XVIII secolo a Parigi da D’Alembert e Cauchy.


\begin{itemize}
\item {} 
\sphinxAtStartPar
Viene richiamato il concetto di \sphinxstylestrong{funzione} di variabile reale a valore reale, \(f: D \in \mathbb{R} \rightarrow \mathbb{R}\), e la sua rappresentazione grafica in un piano cartesiano.

\item {} 
\sphinxAtStartPar
Viene introdotto il concetto di \sphinxstylestrong{limite} per funzioni reali e viene usato per definire le \sphinxstylestrong{funzioni continue}. Vengono quindi presentati alcuni teoremi sulle funzioni continue e sui limiti che ne permettono il calcolo. Vengono presentate le forme indeterminate al finito e all’infinito, e calcolati i \sphinxstyleemphasis{limiti fondamentali}.

\item {} 
\sphinxAtStartPar
Usando i concetti di limite della sezione precedente, viene introdotto il concetto di \sphinxstylestrong{derivata} di una funzione reale, e viene data una sua interpretazione geometrica, legata alla retta tangente al grafico della funzione. Seguono alcune proprietà e teoremi sulle derivate che permettono di valutare le \sphinxstyleemphasis{derivate fondamentali} e combinare questi risultati per il calcolo della derivata di una funzione qualsiasi. Infine viene introdotto il concetto di derivate di ordine superiore, e vengono mostrate alcune applicazioni: ricerca di punti di estremo locale e di flesso nello studio di funzione, ottimizzazione, approssimazione locale tramite sviluppi in serie polinomiali

\item {} 
\sphinxAtStartPar
Viene data la definizione di \sphinxstylestrong{integrale di Riemann} e una sua interpretazione geometrica, legata all’area sottesa dal grafico della funzione. Seguono alcune proprietà degli integrali che permettono di definire l’integrale definito e indefinito, e la primitiva di una funzione. Viene presentato il \sphinxstylestrong{teorema fondamentale del calcolo infinitesimale}, che permette di riconoscere l’operazione di integrazione come inversa dell’integrazione. Usando questo risultato, vengono valutati gli \sphinxstyleemphasis{integrali fondamentali}; poche regole di integrazione permettono poi di calcolare l’integrale di funzioni generiche. Infine vengono mostrate alcune applicazioni: … \sphinxstylestrong{todo}

\end{itemize}



\sphinxstepscope


\chapter{Funzioni reali a variabile reale, \protect\( f: \mathbb{R} \rightarrow \mathbb{R} \protect\)}
\label{\detokenize{ch/infinitesimal_calculus/real_functions:funzioni-reali-a-variabile-reale-f-mathbb-r-rightarrow-mathbb-r}}\label{\detokenize{ch/infinitesimal_calculus/real_functions:infinitesimal-calculus-real-functions}}\label{\detokenize{ch/infinitesimal_calculus/real_functions::doc}}

\section{Definizione}
\label{\detokenize{ch/infinitesimal_calculus/real_functions:definizione}}\label{\detokenize{ch/infinitesimal_calculus/real_functions:infinitesimal-calculus-real-functions-def}}

\section{Rappresentazione grafica}
\label{\detokenize{ch/infinitesimal_calculus/real_functions:rappresentazione-grafica}}\label{\detokenize{ch/infinitesimal_calculus/real_functions:infinitesimal-calculus-real-functions-graph}}
\sphinxstepscope


\chapter{Limiti}
\label{\detokenize{ch/infinitesimal_calculus/limits:limiti}}\label{\detokenize{ch/infinitesimal_calculus/limits:infinitesimal-calculus-limits}}\label{\detokenize{ch/infinitesimal_calculus/limits::doc}}

\section{Cenni di topologia per il calcolo}
\label{\detokenize{ch/infinitesimal_calculus/limits:cenni-di-topologia-per-il-calcolo}}
\sphinxAtStartPar
\sphinxstylestrong{TODO} \sphinxstyleemphasis{Punto di accumulazione e punto isolato, intorno, insiemi aperti e chiusi, limsup/liminf, max/min,…}


\section{Definizione di limite}
\label{\detokenize{ch/infinitesimal_calculus/limits:definizione-di-limite}}


\sphinxAtStartPar
\sphinxstylestrong{Limite finito al finito}
\begin{equation*}
\begin{split}\forall \varepsilon > 0 \quad \exists U_{x_0,\delta} \quad {t.c.} \quad |f(x) - L| < \varepsilon \quad \forall x \in U_{x_0, \delta} \backslash \{x_0\}\end{split}
\end{equation*}
\sphinxAtStartPar
dove la condizione sull’intorno di un punto \(x_0\) al finito per funzioni reali può essere riscritta come \(0 < | x - x_0 | <  \delta\) per un intorno simmetrico del punto \(x_0\).

\sphinxAtStartPar
\sphinxstylestrong{Limite infinito al finito}
\begin{equation*}
\begin{split}\forall M > 0 \quad \exists U_{x_0,\delta} \quad {t.c.} \quad |f(x)| > M \quad \forall x \in U_{x_0, \delta} \backslash \{x_0\}\end{split}
\end{equation*}
\sphinxAtStartPar
dove la condizione sull’intorno di un punto \(x_0\) al finito per funzioni reali può essere riscritta come \(0 < | x - x_0 | <  \delta\) per un intorno simmetrico del punto \(x_0\). Se \(f(x) > M\) allora il limite tende a \(+\infty\), se \(f(x) < -M\) allora il limite tende a \(-\infty\).

\sphinxAtStartPar
\sphinxstylestrong{Limite finito all’infinito}
\begin{equation*}
\begin{split}\forall \varepsilon > 0 \quad \exists U_{\mp\infty,R} \quad {t.c.} \quad |f(x) - L| < \varepsilon \quad \forall x \in U_{\mp\infty, R}\end{split}
\end{equation*}
\sphinxAtStartPar
dove la condizione sull’intorno di un punto all’infinito per funzioni reali può essere riscritta come \(x < R\) per un intorno di \(-\infty\) o \(x > R\) per un intorno di \(+\infty\).

\sphinxAtStartPar
\sphinxstylestrong{Limite infinito all’infinito}
\begin{equation*}
\begin{split}\forall M > 0 \quad \exists U_{\mp \infty, R} \quad {t.c.} \quad |f(x)| > M \quad \forall x \in U_{\mp \infty, R}\end{split}
\end{equation*}
\sphinxAtStartPar
dove la condizione sull’intorno di un punto all’infinito per funzioni reali può essere riscritta come \(x < R\) per un intorno di \(-\infty\) o \(x > R\) per un intorno di \(+\infty\). Se \(f(x) > M\) allora il limite tende a \(+\infty\), se \(f(x) < -M\) allora il limite tende a \(-\infty\).


\section{Funzioni continue}
\label{\detokenize{ch/infinitesimal_calculus/limits:funzioni-continue}}

\subsection{Definizione}
\label{\detokenize{ch/infinitesimal_calculus/limits:definizione}}
\sphinxAtStartPar
Una funzione reale \(f: D \in \mathbb{R} \rightarrow \mathbb{R}\) è continua in un punto \(x_0 \in D\) se la funzione è definita nel punto, se esiste il limite della fuzione e coincide con il valore della funzione
\begin{equation*}
\begin{split}\lim_{x \rightarrow x_0} f(x) = f(x_0) \ .\end{split}
\end{equation*}
\sphinxAtStartPar
Una funzione reale è continua in un dominio \sphinxstylestrong{TODO} \sphinxstylestrong{o insieme?} se è continua in ogni punto del dominio.


\subsection{Teoremi}
\label{\detokenize{ch/infinitesimal_calculus/limits:teoremi}}

\subsubsection{Teorema di Weierstrass}
\label{\detokenize{ch/infinitesimal_calculus/limits:teorema-di-weierstrass}}
\sphinxAtStartPar
\sphinxstylestrong{Enunciato.} Data una funzione reale continua \(f: [a,b] \rightarrow \mathbb{R}\) definita sull’intervallo chiuso \([a,b]\). Allora la funzione \(f(x)\) ammette un punto di massimo assoluto e un punto di minimo assoluto nell’intevallo \([a,b]\).


\subsubsection{Teorema della permanenza del segno}
\label{\detokenize{ch/infinitesimal_calculus/limits:teorema-della-permanenza-del-segno}}
\sphinxAtStartPar
\sphinxstylestrong{Enunciato.} \(f: D \rightarrow \mathbb{R}\) continua, se \(f(x_0) > 0\) allora \(\exists U_{x_0}\) t.c. \(f(x) > 0\) per \(\forall x \in U_{x_0} \cap D\).

\sphinxAtStartPar
\sphinxstylestrong{TODO} \sphinxstyleemphasis{Per ora, enunciato “qualitativo”}


\subsubsection{Teorema dei valori intermedi}
\label{\detokenize{ch/infinitesimal_calculus/limits:teorema-dei-valori-intermedi}}
\sphinxAtStartPar
\sphinxstylestrong{Enunciato.} \(f: [a,b] \rightarrow \mathbb{R}\) continua, allora \(f(x)\) assume tutti i valori compresi tra \(f(a)\) e \(f(b)\), cioè (assumendo \(f(a) < f(b)\)) per \(\forall y \in (f(a), f(b)) \ x_0 \in (a,b) \ \text{t.c..} \ f(x_0) = y\).

\sphinxAtStartPar
\sphinxstylestrong{TODO} \sphinxstyleemphasis{Per ora, enunciato “qualitativo”}


\section{Teoremi sui limiti}
\label{\detokenize{ch/infinitesimal_calculus/limits:teoremi-sui-limiti}}

\subsection{Operazioni coi limiti}
\label{\detokenize{ch/infinitesimal_calculus/limits:operazioni-coi-limiti}}
\sphinxAtStartPar
Dato un numero reale \(c \in \mathbb{R}\) e i limiti \(\lim_{x \rightarrow x_0} f(x) = L_1\), \(\lim_{x \rightarrow x_0} g(x) = L_2\)
\begin{equation*}
\begin{split}\begin{aligned}
 & \lim_{x \rightarrow x_0} \big( c \cdot f(x) \big) = c \, L_1 \\
 & \lim_{x \rightarrow x_0} \big( f(x) \mp g(x) \big) = L_1 + L_2 \\
 & \lim_{x \rightarrow x_0} \big( f(x) \cdot g(x) \big) = L_1 \cdot L_2 \\
 & \lim_{x \rightarrow x_0} \frac{ f(x) }{ g(x) } = \frac{L_1}{L_2} \quad , \quad \text{se $L_2 \ne 0$}  \\
\end{aligned}\end{split}
\end{equation*}

\subsubsection{Limiti infiniti e infinitesimi}
\label{\detokenize{ch/infinitesimal_calculus/limits:limiti-infiniti-e-infinitesimi}}\begin{equation*}
\begin{split}\begin{aligned}
 &  \lim_{x \rightarrow x_0}f(x) \rightarrow \mp \infty \ , c > 0 \quad : \qquad \lim_{x \rightarrow x_0} c \cdot f(x) = \mp \infty \\
 & \dots \\
\end{aligned}\end{split}
\end{equation*}

\subsubsection{Forme indeterminate}
\label{\detokenize{ch/infinitesimal_calculus/limits:forme-indeterminate}}\begin{equation*}
\begin{split}+\infty-\infty \quad , \quad 0 \cdot \mp \infty \quad , \quad \frac{\mp \infty}{\mp \infty} \quad , \quad \frac{0}{0}\end{split}
\end{equation*}

\subsection{Teorema del confronto}
\label{\detokenize{ch/infinitesimal_calculus/limits:teorema-del-confronto}}

\subsection{Teorema di de l’Hopital}
\label{\detokenize{ch/infinitesimal_calculus/limits:teorema-di-de-l-hopital}}
\sphinxAtStartPar
\sphinxstylestrong{TODO.} Si rimanda alla sezione del {\hyperref[\detokenize{ch/infinitesimal_calculus/derivatives:infinitesimal-calculus-derivatives-thm-hopital}]{\sphinxcrossref{\DUrole{std,std-ref}{teorema di de l’Hopital}}}} nel capitolo sulle {\hyperref[\detokenize{ch/infinitesimal_calculus/derivatives:infinitesimal-calculus-derivatives}]{\sphinxcrossref{\DUrole{std,std-ref}{derivate}}}}.


\section{Limiti fondamentali}
\label{\detokenize{ch/infinitesimal_calculus/limits:limiti-fondamentali}}\label{\detokenize{ch/infinitesimal_calculus/limits:infinitesimal-calculus-limits-fund}}
\sphinxAtStartPar
Limiti di funzioni continue
\begin{equation*}
\begin{split} \lim_{x \rightarrow 0} \frac{(1+x)^a - 1}{x} = a \end{split}
\end{equation*}\begin{equation*}
\begin{split} \lim_{x \rightarrow 0} \frac{\sin x}{x} = 1 \end{split}
\end{equation*}\begin{equation*}
\begin{split} \lim_{x \rightarrow 0} \frac{1 - \cos^2 x}{x^2} = \frac{1}{2} \end{split}
\end{equation*}\begin{equation*}
\begin{split} \lim_{x \rightarrow +\infty} \Big( 1 + \frac{1}{x} \Big)^x = e \end{split}
\end{equation*}\begin{equation*}
\begin{split} \lim_{x \rightarrow 0} \frac{e^x - 1}{x}= 1 \end{split}
\end{equation*}\begin{equation*}
\begin{split} \lim_{x \rightarrow 0} \frac{e^x}{1+x}= 1 \end{split}
\end{equation*}\begin{equation*}
\begin{split} \lim_{x \rightarrow 0} \frac{\ln (1+x)}{x} = 1 \end{split}
\end{equation*}
\sphinxAtStartPar
Una volta compresa l’operazione di derivazione e di sviluppo in serie, si può rivisitare i limiti notevoli \sphinxstylestrong{todo}

\sphinxAtStartPar
Limiti di successioni. \sphinxstylestrong{Formula di Sterling}
\begin{equation*}
\begin{split}n! \sim \left(\frac{n}{e} \right)^n \qquad \text{per $n \in \mathbb{N} \rightarrow +\infty$}\end{split}
\end{equation*}
\sphinxAtStartPar
o
\begin{equation*}
\begin{split}\ln n! \sim n \ln n - n  \qquad \text{per $n \in \mathbb{N} \rightarrow +\infty$}\end{split}
\end{equation*}

\section{Infiniti e infinitesimi}
\label{\detokenize{ch/infinitesimal_calculus/limits:infiniti-e-infinitesimi}}
\sphinxstepscope


\chapter{Derivate}
\label{\detokenize{ch/infinitesimal_calculus/derivatives:derivate}}\label{\detokenize{ch/infinitesimal_calculus/derivatives:infinitesimal-calculus-derivatives}}\label{\detokenize{ch/infinitesimal_calculus/derivatives::doc}}

\section{Definizione}
\label{\detokenize{ch/infinitesimal_calculus/derivatives:definizione}}\label{\detokenize{ch/infinitesimal_calculus/derivatives:infinitesimal-calculus-derivatives-def}}
\sphinxAtStartPar
\sphinxstylestrong{Rapporto incrementale.} Il rapporto incrementale di una funzione reale nel punto \(x\) viene definito come il rapporto tra la differenza dei valori della funzione e la differenza del valore della variabile indipendente
\begin{equation}\label{equation:ch/infinitesimal_calculus/derivatives:infinitesimal-calculus:derivatives:def_delta}
\begin{split}R[f(\cdot), x, a] := \dfrac{f(x+a)-f(x)}{a} \ .\end{split}
\end{equation}
\sphinxAtStartPar
\sphinxstylestrong{Derivata.} La derivata di una funzione reale in un punto \(x\) viene definita come il limite del rapporto incrementale, per l’incremento della variabile indipendente che tende a zero,
\begin{equation}\label{equation:ch/infinitesimal_calculus/derivatives:infinitesimal-calculus:derivatives:def}
\begin{split}f'(x) = \dfrac{d f}{d x}(x) := \lim_{a \rightarrow 0} \dfrac{f(x+a)-f(x)}{a} \ .\end{split}
\end{equation}
\sphinxAtStartPar
\sphinxstylestrong{todo} \sphinxstyleemphasis{In generale, la derivata di una funzione reale è un’altra funzione reale.}


\section{Regole di derivazione}
\label{\detokenize{ch/infinitesimal_calculus/derivatives:regole-di-derivazione}}\label{\detokenize{ch/infinitesimal_calculus/derivatives:infinitesimal-calculus-derivatives-rules}}
\sphinxAtStartPar
Usando la definizione \eqref{equation:ch/infinitesimal_calculus/derivatives:infinitesimal-calculus:derivatives:def} di derivata e le proprietà dei limiti, è possibile dimostrare le seguenti proprietà
\begin{itemize}
\item {} 
\sphinxAtStartPar
linearità

\end{itemize}
\begin{equation}\label{equation:ch/infinitesimal_calculus/derivatives:infinitesimal-calculus:derivatives:rules:linearity}
\begin{split}\big( a \, f(x) + b \, g(x) \big)' = a \, f'(x) + b \, g'(x)\end{split}
\end{equation}\begin{itemize}
\item {} 
\sphinxAtStartPar
derivata del prodotto di funzioni

\end{itemize}
\begin{equation}\label{equation:ch/infinitesimal_calculus/derivatives:infinitesimal-calculus:derivatives:rules:product}
\begin{split}\Bigl( f(x) g(x) \Bigr)' = f'(x) g(x) + f(x) g'(x)\end{split}
\end{equation}\begin{itemize}
\item {} 
\sphinxAtStartPar
derivata del rapporto di funzioni

\end{itemize}
\begin{equation}\label{equation:ch/infinitesimal_calculus/derivatives:infinitesimal-calculus:derivatives:rules:division}
\begin{split}\Big( \frac{f(x)}{g(x)} \Big)' = \frac{f'(x) g(x) - f(x) g'(x)}{g^2(x)}\end{split}
\end{equation}\begin{itemize}
\item {} 
\sphinxAtStartPar
derivata della funzione composta

\end{itemize}
\begin{equation}\label{equation:ch/infinitesimal_calculus/derivatives:infinitesimal-calculus:derivatives:rules:composite}
\begin{split}\frac{d}{dx} f\big( g(x) \big) = \frac{d f}{dy}\Big|_{y=g(x)} \dfrac{d g}{d x}\Big|_{x}\end{split}
\end{equation}\begin{itemize}
\item {} 
\sphinxAtStartPar
derivata della funzione inversa, \(y = f(x)\), \(x = f^{-1}(y)\)

\end{itemize}
\begin{equation}\label{equation:ch/infinitesimal_calculus/derivatives:infinitesimal-calculus:derivatives:rules:inverse}
\begin{split} \dfrac{d f^{-1}}{d y}\bigg|_{y = f(x)} = \dfrac{1}{ \dfrac{d y}{d x}\bigg|_{x}} \ .\end{split}
\end{equation}\subsubsection*{Dimostrazione della linearità dell’operazione di derivazione}

\sphinxAtStartPar
\sphinxstylestrong{todo}
\subsubsection*{Dimostrazione della regola del prodotto}

\sphinxAtStartPar
\sphinxstylestrong{todo}
\subsubsection*{Dimostrazione della regola del quoziente}

\sphinxAtStartPar
\sphinxstylestrong{todo}
\subsubsection*{Dimostrazione della regola della funzione composta}

\sphinxAtStartPar
\sphinxstylestrong{todo}
\subsubsection*{Dimostrazione della regola della funzione inversa}

\sphinxAtStartPar
Si usa la regola \eqref{equation:ch/infinitesimal_calculus/derivatives:infinitesimal-calculus:derivatives:rules:composite} di derivazione della funzione composta applicata alla relazione
\begin{equation*}
\begin{split}x = f^{-1} \left( f(x) \right)\end{split}
\end{equation*}
\sphinxAtStartPar
che caratterizza la funzione inversa \(f^{-1}\). Derivando entrambi i termini della relazione rispetto alla variabile indipendente \(x\) si ottiene
\begin{equation*}
\begin{split}1 = \dfrac{d f^{-1}}{d y}\bigg|_{y = f(x)} \, \dfrac{d f(x)}{d x} \ ,\end{split}
\end{equation*}
\sphinxAtStartPar
dalla quale segue immediatamente la regola di derivazione della funzione inversa
\begin{equation*}
\begin{split} \dfrac{d f^{-1}}{d y}\bigg|_{y = f(x)} = \dfrac{1}{ \dfrac{d y}{d x}\bigg|_{x}} \ .\end{split}
\end{equation*}

\section{Teoremi}
\label{\detokenize{ch/infinitesimal_calculus/derivatives:teoremi}}\label{\detokenize{ch/infinitesimal_calculus/derivatives:infinitesimal-calculus-derivatives-thm}}\label{ch/infinitesimal_calculus/derivatives:theorem-0}
\begin{sphinxadmonition}{note}{Theorem 1 (Teorema di Fermat)}



\sphinxAtStartPar
Data la funzione \(f: (a,b) \rightarrow \mathbb{R}\) derivabile nel punto di estremo locale \(x_0 \in (a,b)\), allora \(f'(x_0) = 0\).
\end{sphinxadmonition}
\subsubsection*{Dimostrazione}

\sphinxAtStartPar
\sphinxstylestrong{todo}


\label{ch/infinitesimal_calculus/derivatives:theorem-1}
\begin{sphinxadmonition}{note}{Theorem 2 (Teorema di Rolle)}



\sphinxAtStartPar
Data la funzione \(f: [a,b] \rightarrow \mathbb{R}\) continua e derivabile in ogni punto dell’intervallo \((a,b)\) con \(f(a) = f(b)\), allora esiste un valore \(c \in (a,b)\) in cui \(f'(c) = 0\).
\end{sphinxadmonition}
\subsubsection*{Dimostrazione}

\sphinxAtStartPar
\sphinxstylestrong{todo}
\label{ch/infinitesimal_calculus/derivatives:theorem-2}
\begin{sphinxadmonition}{note}{Theorem 3 (Teorema di Cauchy)}



\sphinxAtStartPar
Date le funzioni \(f, g: [a,b] \rightarrow \mathbb{R}\) continue e derivabili in ogni punto dell’intervallo \((a,b)\) con \(f(a) = f(b)\), allora esiste un valore \(c \in (a,b)\) tale che
\begin{equation*}
\begin{split}f(b) - f(a) = (b -  a) f'(c) \ .\end{split}
\end{equation*}\end{sphinxadmonition}
\subsubsection*{Dimostrazione}

\sphinxAtStartPar
\sphinxstylestrong{todo}
\label{ch/infinitesimal_calculus/derivatives:theorem-3}
\begin{sphinxadmonition}{note}{Theorem 4 (Theorema di Lagrange)}



\sphinxAtStartPar
Date le funzioni \(f, g: [a,b] \rightarrow \mathbb{R}\) continue e derivabili in ogni punto dell’intervallo \((a,b)\) con \(f(a) = f(b)\), allora esiste un valore \(c \in (a,b)\) tale che
\end{sphinxadmonition}
\subsubsection*{Dimostrazione}

\sphinxAtStartPar
\sphinxstylestrong{todo} Usando il teorema di Cauhcy con \(g(x) = x\).
\label{ch/infinitesimal_calculus/derivatives:theorem-4}
\begin{sphinxadmonition}{note}{Theorem 5 (Teorema di de l’Hopital (Bernoulli))}



\sphinxAtStartPar
Siano \(f(x), g(x): [a,b] \rightarrow \mathbb{R}\) funzioni reali di variabile reale continue in \([a,b]\) e derivabili in \((a,b) \backslash \{ x_0 \}\).
\begin{equation*}
\begin{split}\begin{aligned}
  & \lim_{x \rightarrow x_0} f(x) = \lim_{x \rightarrow x_0} g(x) = 0 \qquad \text{oppure}
  & \lim_{x \rightarrow x_0}|f(x)|= \lim_{x \rightarrow x_0}|g(x)|= \infty
\end{aligned}\end{split}
\end{equation*}
\sphinxAtStartPar
Se esiste
\begin{equation*}
\begin{split} \lim_{x \rightarrow x_0} \frac{f'(x)}{g'(x)} = L \quad \text{finito}\end{split}
\end{equation*}
\sphinxAtStartPar
allora
\begin{equation*}
\begin{split} \lim_{x \rightarrow x_0} \frac{f(x)}{g(x)} = L \ . \end{split}
\end{equation*}
\sphinxAtStartPar
\sphinxstylestrong{todo} Controllare l’enunciato
\end{sphinxadmonition}
\subsubsection*{Dimostrazione}

\sphinxAtStartPar
\sphinxstylestrong{todo}

\sphinxAtStartPar
\sphinxstylestrong{Oss.} Il teorema di de l’Hopital può essere applicato anche in successione, più di una volta, fermandosi al primo rapporto di derivate dello stesso ordine che non produce una forma indeterminata.


\section{Derivate fondamentali}
\label{\detokenize{ch/infinitesimal_calculus/derivatives:derivate-fondamentali}}\label{\detokenize{ch/infinitesimal_calculus/derivatives:infinitesimal-calculus-derivatives-fund}}
\sphinxAtStartPar
Usando i {\hyperref[\detokenize{ch/infinitesimal_calculus/limits:infinitesimal-calculus-limits-fund}]{\sphinxcrossref{\DUrole{std,std-ref}{limiti fondamentali}}}}, vengono calcolate le derivate fondamentali, che a loro volta permettono il calcolo degli {\hyperref[\detokenize{ch/infinitesimal_calculus/integrals:infinitesimal-calculus-integrals-fund}]{\sphinxcrossref{\DUrole{std,std-ref}{integrali fondamentali}}}}. Le derivate fondamentali e la loro combinazione con le {\hyperref[\detokenize{ch/infinitesimal_calculus/derivatives:infinitesimal-calculus-derivatives-rules}]{\sphinxcrossref{\DUrole{std,std-ref}{regole di derivazione}}}} permettono la derivazione di funzioni generiche. Le derivate fondamentali sono:
\begin{equation}\label{equation:ch/infinitesimal_calculus/derivatives:infinitesimal-calculus:derivatives:fund}
\begin{split}\begin{aligned}
f(x) & = x^n    \qquad & \qquad f'(x) & = n x^{n-1}   \\ 
f(x) & = e^x    \qquad & \qquad f'(x) & = e^x         \\ 
f(x) & = \ln x  \qquad & \qquad f'(x) & = \frac{1}{x} \\ 
f(x) & = \sin x \qquad & \qquad f'(x) & = \cos x      \\ 
f(x) & = \cos x \qquad & \qquad f'(x) & =-\sin x         
\end{aligned}\end{split}
\end{equation}\subsubsection*{Dimostrazione di \protect\(\ (x^n)'\protect\)}

\sphinxAtStartPar
Usando la formua binomiale \$\((x + \varepsilon)^n = x^n + n x^{n-1} \varepsilon + f(\varepsilon^2, \varepsilon^3, \dots)\)\$ \sphinxstylestrong{todo} \sphinxstyleemphasis{aggiungere riferimento},
\begin{equation*}
\begin{split}\begin{aligned}
  \dfrac{d}{dx} x^n
  & = \lim_{\varepsilon \rightarrow 0}  \dfrac{(x+\varepsilon)^{n} - x^n}{\varepsilon} = \\
  & = \lim_{\varepsilon \rightarrow 0}  \dfrac{x^n + n x^{n-1} \varepsilon + o(\varepsilon) - x^n}{\varepsilon} = \\
  & = \lim_{\varepsilon \rightarrow 0}  \left( n x^{n-1} + O(\varepsilon) \right) = \\
  & = n x^{n-1} \ .
\end{aligned}\end{split}
\end{equation*}\subsubsection*{Dimostrazione di \protect\(\ (e^x)'\protect\)}

\sphinxAtStartPar
Usando le proprietà della funzione esponenziale e il limite \(e^{\varepsilon} - 1 \sim \varepsilon\) per \(\varepsilon \rightarrow 0\)
\begin{equation*}
\begin{split}\begin{aligned}
  \dfrac{d}{dx} e^x       
  & = \lim_{\varepsilon \rightarrow 0}  \dfrac{e^{x+\varepsilon} - e^x}{\varepsilon} = \\
  & = \lim_{\varepsilon \rightarrow 0}  \dfrac{e^x \left( e^{\varepsilon} - 1 \right)}{\varepsilon} = \\
  & = e^x \lim_{\varepsilon \rightarrow 0}  \dfrac{\varepsilon + o(\varepsilon)}{\varepsilon} = \\
  & = e^x \lim_{\varepsilon \rightarrow 0}  \left( 1 + O(\varepsilon) \right) = \\
  & = e^x \ .
\end{aligned}\end{split}
\end{equation*}\subsubsection*{Dimostrazione di \protect\(\ (\ln x)'\protect\)}

\sphinxAtStartPar
Usando le proprietà della funzione logaritmo naturale e il limite \(\ln(1 + \varepsilon) \sim \varepsilon\) per \(\varepsilon \rightarrow 0\), per \(x > 0\)
\begin{equation*}
\begin{split}\begin{aligned}
  \dfrac{d}{dx} \ln x       
  & = \lim_{\varepsilon \rightarrow 0}  \dfrac{\ln(x+\varepsilon) - \ln x}{\varepsilon} = \\
  & = \lim_{\varepsilon \rightarrow 0}  \dfrac{\ln \left(1 + \frac{\varepsilon}{x} \right)}{\varepsilon} = \\
  & = \lim_{\varepsilon \rightarrow 0}  \dfrac{\frac{\varepsilon}{x} + o(\varepsilon)}{\varepsilon} = \\
  & = \lim_{\varepsilon \rightarrow 0}  \left( \frac{1}{x} + O(\varepsilon) \right) = \\
  & = \frac{1}{x} \ .
\end{aligned}\end{split}
\end{equation*}\subsubsection*{Dimostrazione di \protect\(\ (\sin x)'\protect\)}

\sphinxAtStartPar
Usando le formule di somma delle funzioni armoniche, \sphinxstylestrong{todo} ref, e gli infinitesimi delle funzioni \(\sin \varepsilon \sim \varepsilon\), \(\cos \varepsilon \sim 1 - \frac{\varepsilon^2}{2}\) per \(\varepsilon \rightarrow 0\),
\begin{equation*}
\begin{split}\begin{aligned}
  \dfrac{d}{dx} \sin(x) 
  & = \lim_{\varepsilon \rightarrow 0}  \dfrac{\sin(x+\varepsilon) - \sin x}{\varepsilon} = \\
  & = \lim_{\varepsilon \rightarrow 0} \dfrac{\sin x \cos \varepsilon + \cos x \sin \varepsilon - \sin x}{\varepsilon} = \\
  & = \lim_{\varepsilon \rightarrow 0} \dfrac{\sin x \left( 1 - \frac{\varepsilon^2}{2} \right) + \varepsilon \, \cos x - \sin x}{\varepsilon} = \\
  & = \lim_{\varepsilon \rightarrow 0} \left( \cos x + O(\varepsilon) \right) = \\
  & = \cos x \ .
\end{aligned}\end{split}
\end{equation*}\subsubsection*{Dimostrazione di \protect\(\ (\cos x)'\protect\)}

\sphinxAtStartPar
Usando le formule di somma delle funzioni armoniche, \sphinxstylestrong{todo} ref, e gli infinitesimi delle funzioni \(\sin \varepsilon \sim \varepsilon\), \(\cos \varepsilon \sim 1 - \frac{\varepsilon^2}{2}\) per \(\varepsilon \rightarrow 0\),
\begin{equation*}
\begin{split}\begin{aligned}
  \dfrac{d}{dx} \cos(x) 
  & = \lim_{\varepsilon \rightarrow 0}  \dfrac{\cos(x+\varepsilon) - \cos x}{\varepsilon} = \\
  & = \lim_{\varepsilon \rightarrow 0} \dfrac{\cos x \cos \varepsilon - \sin x \sin \varepsilon - \sin x}{\varepsilon} = \\
  & = \lim_{\varepsilon \rightarrow 0} \dfrac{\cos x \left( 1 - \frac{\varepsilon^2}{2} \right) - \varepsilon \, \sin x - \cos x}{\varepsilon} = \\
  & = \lim_{\varepsilon \rightarrow 0} \left( - \sin x + O(\varepsilon) \right) = \\
  & = - \sin x \ .
\end{aligned}\end{split}
\end{equation*}

\section{Derivate di ordine superiore}
\label{\detokenize{ch/infinitesimal_calculus/derivatives:derivate-di-ordine-superiore}}\label{\detokenize{ch/infinitesimal_calculus/derivatives:infinitesimal-calculus-derivatives-higher}}
\sphinxAtStartPar
Nel calcolo delle derivate di ordine superiore non c’è nulla di speciale: una volta che si è in grado di calcolare la derivata di una funzione reale, la derivata di ordine \(n\) viene calcolata applicando \(n\) volte l’operatore derivata alla funzione.


\section{Applicazioni}
\label{\detokenize{ch/infinitesimal_calculus/derivatives:applicazioni}}\label{\detokenize{ch/infinitesimal_calculus/derivatives:infinitesimal-calculus-derivatives-applications}}

\subsection{Espansioni in serie di Taylor e MacLaurin}
\label{\detokenize{ch/infinitesimal_calculus/derivatives:espansioni-in-serie-di-taylor-e-maclaurin}}\label{\detokenize{ch/infinitesimal_calculus/derivatives:infinitesimal-calculus-derivatives-taylor}}
\sphinxAtStartPar
Le espansioni in serie di Taylor e di MacLaurin sono serie polinomiali che forniscono un’\sphinxstylestrong{approssimazione locale} di una funzione, \sphinxstyleemphasis{valida nell’intorno} (\sphinxstylestrong{todo} valutare questa espressione) di un punto.

\sphinxAtStartPar
La \sphinxstylestrong{serie di Taylor} della funzione \(f(x)\) in un intervallo centrato in \(x_0\) è la serie
\begin{equation*}
\begin{split}\begin{aligned}
  T[f(x); x_0] & = \sum_{n=0}^{\infty} \dfrac{f^{(n)(x_0)}}{n!} (x-x_0)^n = \\
       & = f(x_0) + f'(x_0) (x-x_0) + \dfrac{f''(x_0)^2}{2!} (x-x_0)^2 + \dots \ .
\end{aligned}\end{split}
\end{equation*}
\sphinxAtStartPar
La serie di MacLaurin è la serie di Taylor centrata in \(x_0 = 0\).

\sphinxAtStartPar
La serie di Taylor troncata al \(n\)\sphinxhyphen{}esimo termine fornisce un’approssimazione locale della funzione \(f(x)\) di ordine \(n\), nel senso definito dal seguente teorema.
\label{ch/infinitesimal_calculus/derivatives:theorem-5}
\begin{sphinxadmonition}{note}{Theorem 6 (Approssimazione locale)}


\begin{equation*}
\begin{split}\lim_{x \rightarrow x_0} \frac{f(x) - T[f(x); x_0]}{x^n} = f^{(n)}(x_0) \ , \end{split}
\end{equation*}\begin{equation*}
\begin{split}f(x) = T[f(x); x_0] + o(x^n) \quad \text{ per } \quad x \rightarrow x_0\end{split}
\end{equation*}\end{sphinxadmonition}
\subsubsection*{Dimostrazione}

\sphinxAtStartPar
Usando il teorema di de l’Hopital, fino a quando il rapporto non è una forma indeterminata
\begin{equation*}
\begin{split}\begin{aligned}
  \lim_{x \rightarrow x_0} \frac{f(x) - T[f(x); x_0]}{x^n}
  & = \lim_{x \rightarrow x_0} \dfrac{f(x) - f(x_0) + f'(x_0) (x-x_0) + \frac{f''(x_0)}{2!} (x-x_0)^2 + \dots \frac{f^{(n)}(x_0)}{n!}(x-x_0)^n}{x^n} = \text{(H)} = \\
  & = \lim_{x \rightarrow x_0} \dfrac{f'(x) - f'(x_0) + \frac{f''(x_0)}{1!} (x-x_0) + \dots \frac{f^{(n)}(x_0)}{(n-1)!}(x-x_0)^{n-1}}{n \, x^{n-1}} = \text{(H)} = \\
  & = \lim_{x \rightarrow x_0} \dfrac{f''(x) - f''(x_0) + \dots \frac{f^{(n)}(x_0)}{(n-2)!}(x-x_0)^{n-2}}{n \, (n-1) \, x^{n-1}} = \text{(H)} =\\
  & = \dots \\
  & = \lim_{x \rightarrow x_0} \dfrac{f^{(n)}(x) - f^{(n)}(x_0)}{n!} =  0 \ ,
\end{aligned}\end{split}
\end{equation*}
\sphinxAtStartPar
si dimostra che il numeratore è un infinitesimo del denominatore. Usando la notazione dell’\sphinxstyleemphasis{”o piccolo”} per gli infinitesimi si può quindi scrivere l’approssimazione locale come:
\begin{equation*}
\begin{split}f(x) - T[f(x), x_0] = o\left((x-x_0)^n\right) \ ,\end{split}
\end{equation*}
\sphinxAtStartPar
o in maniera equivalente
\$\(f(x) = T[f(x), x_0] + o\left((x-x_0)^n\right) \ .\)\$


\subsubsection{Esempi}
\label{\detokenize{ch/infinitesimal_calculus/derivatives:esempi}}
\sphinxAtStartPar
La serie di MacLaurin per le funzioni interessate nei {\hyperref[\detokenize{ch/infinitesimal_calculus/limits:infinitesimal-calculus-limits-fund}]{\sphinxcrossref{\DUrole{std,std-ref}{limiti notevoli}}}} forniscono approssimazioni locali di ordine maggiore per \(x \rightarrow 0\),
\begin{equation*}
\begin{split}\begin{aligned}
 \cos(x) & = 1 - \frac{x^2}{2!} + \frac{x^4}{4!} + o(x^5) \\
 \sin(x) & = x - \frac{x^3}{3!} + \frac{x^5}{5!} + o(x^6) \\
 \ln (1+x) & = x - \frac{x^2}{2} + \frac{x^3}{3} + o(x^3) \\
 (1+x)^a & = 1 + a x + a(a-1) \frac{x^2}{2} + o(x^2) \\
 e^x     & = 1 + x + \frac{x^2}{2!} + \frac{x^3}{3!} + \frac{x^4}{4!} + \frac{x^5}{5!} + o(x^5)
\end{aligned}\end{split}
\end{equation*}
\sphinxAtStartPar
\sphinxstylestrong{todo} \sphinxstyleemphasis{Dimostrare la convergenza delle serie. Convergenza puntuale, convergenza uniforme (in un insieme di convergenza, di solito centrato in un punto e le cui dimensioni sono definite da un raggio di convergenza)}

\sphinxAtStartPar
\sphinxstylestrong{Rivisitazione limiti notevoli}
Per \(x \rightarrow 0\)
\begin{equation*}
\begin{split}(1+x)^a - 1 = a \, x + o(x)\end{split}
\end{equation*}\begin{equation*}
\begin{split}\sin x = x +o(x)\end{split}
\end{equation*}\begin{equation*}
\begin{split}1 - \cos x = \frac{1}{2} x^2 + o(x^3)\end{split}
\end{equation*}\begin{equation*}
\begin{split}e^x - 1 = x + o(x)\end{split}
\end{equation*}\begin{equation*}
\begin{split}\ln(1+x) = x + o(x)\end{split}
\end{equation*}
\sphinxAtStartPar
\sphinxstylestrong{Identità di Eulero.} Usando l’espansione in serie di Taylor per l’esponenziale complesso \(e^{ix}\), si ottiene
\begin{equation*}
\begin{split}\begin{aligned}
e^{ix} & = 1 + ix + \frac{(ix)^2}{2!} + \frac{(ix)^3}{3!} + \frac{(ix)^4}{4!} + \frac{x^5}{5!} + o(x^5) = \\
& = \Big( 1 - \frac{x^2}{2!} + \frac{x^4}{4!} \Big) + i \Big( x - \frac{x^3}{3!} + \frac{x^5}{5!} \Big) + o(x^5) = \\
& = \cos x + i \sin x \ .
\end{aligned}\end{split}
\end{equation*}

\subsection{Studio di funzione}
\label{\detokenize{ch/infinitesimal_calculus/derivatives:studio-di-funzione}}

\subsection{Ottimizzazione}
\label{\detokenize{ch/infinitesimal_calculus/derivatives:ottimizzazione}}
\sphinxstepscope


\chapter{Integrali}
\label{\detokenize{ch/infinitesimal_calculus/integrals:integrali}}\label{\detokenize{ch/infinitesimal_calculus/integrals:infinitesimal-calculus-integrals}}\label{\detokenize{ch/infinitesimal_calculus/integrals::doc}}

\section{Definizioni}
\label{\detokenize{ch/infinitesimal_calculus/integrals:definizioni}}\label{\detokenize{ch/infinitesimal_calculus/integrals:infinitesimal-calculus-integrals-def}}
\sphinxAtStartPar
\sphinxstylestrong{Somma di Riemann.} Data una funzione continua \(f: [a,b] \rightarrow \mathbb{R}\) e \(P = \{ x_0, x_1, \dots x_n | a = x_0 < x_1 < \dots < x_n = b \}\) partizione dell’intervallo \([a,b]\), la somma di Riemann viene definita come
\begin{equation}\label{equation:ch/infinitesimal_calculus/integrals:infinitesimal-calculus:integrals:riemann:sum}
\begin{split}\sigma_P = \sum_{k=1}^{n} f(\xi_k) (x_{k} - x_{k-1}) \ ,\end{split}
\end{equation}
\sphinxAtStartPar
con \(\xi_k \in [x_{k-1}, x_k]\).

\sphinxAtStartPar
\sphinxstylestrong{Integrale di Riemann.} Sia \(\Delta x = \max_k (x_{k} - x_{k-1})\), l’integrale definito di Riemann è  il limite per \(\Delta x \rightarrow 0\) della somma di Riemann \(\sigma\)
\begin{equation}\label{equation:ch/infinitesimal_calculus/integrals:infinitesimal-calculus:integrals:riemann:def}
\begin{split}\int_a^b f(x) \ dx = \lim_{\Delta x \rightarrow 0} \sigma_P \ .\end{split}
\end{equation}
\sphinxAtStartPar
\sphinxstylestrong{Osservazione.} Dato l’intervallo \([a,b]\), per \(\Delta x \rightarrow 0\) il numero di intervalli della partizione tende all’infinito, \(n \rightarrow \infty\).


\subsection{Interpretazione geometrica}
\label{\detokenize{ch/infinitesimal_calculus/integrals:interpretazione-geometrica}}
\sphinxAtStartPar
L’integrale definito
\begin{equation*}
\begin{split}\int_{a}^{b} f(x) \, dx \ ,\end{split}
\end{equation*}
\sphinxAtStartPar
corrisponde al valore dell’\sphinxstylestrong{area con segno} tra il grafico della funzione \(y=f(x)\) e l’asse \(x\), per valori di \(x \in [a,b]\). Se la funzione è positiva in un intervallo, il contributo dell’integrale sull’intervallo è positivo; se la funzione è negativa in un intervallo, il contributo dell’integrale sull’intervallo è negativo.


\subsection{Integrale definito}
\label{\detokenize{ch/infinitesimal_calculus/integrals:integrale-definito}}\label{\detokenize{ch/infinitesimal_calculus/integrals:infinitesimal-calculus-integrals-def-definite}}

\subsubsection{Proprietà dell’integrale definito}
\label{\detokenize{ch/infinitesimal_calculus/integrals:proprieta-dell-integrale-definito}}\label{\detokenize{ch/infinitesimal_calculus/integrals:infinitesimal-calculus-integrals-def-definite-prop}}\phantomsection\label{\detokenize{ch/infinitesimal_calculus/integrals:infinitesimal-calculus-integrals-def-indefinite}}
\sphinxAtStartPar
Dalla definizione \eqref{equation:ch/infinitesimal_calculus/integrals:infinitesimal-calculus:integrals:riemann:def} dell’integrale di Riemann seguono immediatamente le seguenti proprietà:
\begin{itemize}
\item {} 
\sphinxAtStartPar
linearità dell’integrale definito

\end{itemize}
\begin{equation}\label{equation:ch/infinitesimal_calculus/integrals:infinitesimal-calculus:integrals:prop:linearity}
\begin{split}\int_a^b \big( \alpha f(x) + \beta g(x) \big) \ dx = \alpha \int_a^b f(x) \ dx + \beta \int_a^b g(x) \ dx \ ,\end{split}
\end{equation}\begin{itemize}
\item {} 
\sphinxAtStartPar
additività sull’intervallo

\end{itemize}
\begin{equation}\label{equation:ch/infinitesimal_calculus/integrals:infinitesimal-calculus:integrals:prop:add}
\begin{split}\int_a^b f(x) \ dx + \int_b^c f(x) \ dx = \int_a^c f(x) \ dx \ ,\end{split}
\end{equation}\begin{itemize}
\item {} 
\sphinxAtStartPar
valore assoluto dell’integrale è minore dell’integrale del valore assoluto

\end{itemize}
\begin{equation}\label{equation:ch/infinitesimal_calculus/integrals:infinitesimal-calculus:integrals:prop:abs}
\begin{split}\left| \int_a^b f(x) \ dx \right| \le \int_a^b | f(x) | \ dx \ ,\end{split}
\end{equation}\begin{itemize}
\item {} 
\sphinxAtStartPar
scambio degli estremi di integrazione

\end{itemize}
\begin{equation}\label{equation:ch/infinitesimal_calculus/integrals:infinitesimal-calculus:integrals:prop:swap}
\begin{split}\int_{x=a}^{b} f(x) dx = - \int_{x=b}^{a} f(x) \, dx\end{split}
\end{equation}

\subsection{Integrale indefinito}
\label{\detokenize{ch/infinitesimal_calculus/integrals:integrale-indefinito}}
\sphinxAtStartPar
Usando la proprietà \eqref{equation:ch/infinitesimal_calculus/integrals:infinitesimal-calculus:integrals:prop:add} di additività sull’intervallo dell’integrale definito,
\begin{equation*}
\begin{split}\int_a^x f(t) \ dt = \int_a^b f(t) \ dt + \int_b^x f(t) \ dt \ , \end{split}
\end{equation*}
\sphinxAtStartPar
si osserva che i due integrali con estremo superiore \(x\) e diverso estremo inferiore differiscono solo per una quantità indipendente da \(x\), \(\int_{a}^{b} f(t) \ dt\). Data la funzione \(f(x)\) e il valore \(a\) come paramtetro, si definisce una funzione di \(x\)
\begin{equation}\label{equation:ch/infinitesimal_calculus/integrals:infinitesimal-calculus:integrals:primi-}
\begin{split}F(x;a) := \int_a^x f(t) \ dt \ .\end{split}
\end{equation}
\sphinxAtStartPar
Usando questa definizione, è immediato dimostrare che l’integrale definito \(\int_{a}^{b} f(t) \ dt\) è uguale alla differenza della funzione \(F(\cdot; b)\) calcolata nei due estremi,
\begin{equation*}
\begin{split}\begin{aligned}
  \int_{a}^{b} f(t) \ dt & = \int_{c}^{b} f(t) dt + \int_{a}^{c} f(t) dt = \\ 
                         & = \int_{c}^{b} f(t) dt - \int_{c}^{a} f(t) dt = \\
                         & = F(b;c) - F(a;c) \ ,
\end{aligned}\end{split}
\end{equation*}
\sphinxAtStartPar
e che questo risultato è indipendente dal valore \(c\), usato come parametro nella definizione della funzione \(F\).

\sphinxAtStartPar
Data una funzione \(f(x)\), le due funzioni \(F(x;a_1)\), \(F(x;a_2)\) differiscono solo di un termine che dipende dai parametri \(a_1\), \(a_2\) ma non dalla variabile indipendente \(x\). La famiglia di funzioni \(F(x;a)\) ottenuta per ogni valore di \(a\) definisce quindi una funzione \(F(x)\) a meno di una costante additiva, la \sphinxstylestrong{funzione primitiva} della funzione \(f(x)\).

\sphinxAtStartPar
L’\sphinxstylestrong{integrale indefinito} di una funzione \(f(x)\) viene definito come,
\begin{equation*}
\begin{split}\int^x f(t) \ dt = F(x) + C \ ,\end{split}
\end{equation*}
\sphinxAtStartPar
dove la costante additiva \(C\) tiene conto dell’arbitrarietà appena discussa.


\section{Teoremi}
\label{\detokenize{ch/infinitesimal_calculus/integrals:teoremi}}\label{\detokenize{ch/infinitesimal_calculus/integrals:infinitesimal-calculus-derivatives-thm}}

\label{ch/infinitesimal_calculus/integrals:theorem-0}
\begin{sphinxadmonition}{note}{Theorem 7 (Teorema della media)}



\sphinxAtStartPar
Sia \(f: [a,b] \in \mathbb{R} \rightarrow \mathbb{R}\) una funzione continua su \([a,b]\), allora esiste \(c \in [a,b]\) tale che
\begin{equation*}
\begin{split}\int_{a}^{b} f(x) dx = (b-a) f(c) \end{split}
\end{equation*}\end{sphinxadmonition}
\subsubsection*{Dimostrazione}

\sphinxAtStartPar
\sphinxstylestrong{todo}


\label{ch/infinitesimal_calculus/integrals:theorem-1}
\begin{sphinxadmonition}{note}{Theorem 8 (Teorema fondamentale del calcolo infinitesimale)}


\begin{equation*}
\begin{split}\dfrac{d}{dx} \int_{a}^{x} f(y) dy = f(x) \end{split}
\end{equation*}\end{sphinxadmonition}
\subsubsection*{Dimostrazione}

\sphinxAtStartPar
\sphinxstylestrong{Dim.} Usando la {\hyperref[\detokenize{ch/infinitesimal_calculus/derivatives:infinitesimal-calculus-derivatives-def}]{\sphinxcrossref{\DUrole{std,std-ref}{definizione di derivata}}}}, le {\hyperref[\detokenize{ch/infinitesimal_calculus/integrals:infinitesimal-calculus-integrals-def-definite-prop}]{\sphinxcrossref{\DUrole{std,std-ref}{proprietà dell’integrale definito}}}} e il {\hyperref[\detokenize{ch/infinitesimal_calculus/integrals:infinitesimal-calculus-derivatives-thm-avg}]{\sphinxcrossref{\DUrole{std,std-ref}{teorema della media}}}},
\begin{equation*}
\begin{split}\begin{aligned}
\dfrac{d}{dx} \int_{a}^x f(y) dy & = \lim_{\varepsilon \rightarrow 0 }\frac{1}{\varepsilon} \Big[ \int_{a}^{x+\varepsilon} f(y) dy - \int_{a}^{x} f(y) dy \Big] = \\
& = \lim_{\varepsilon \rightarrow 0 }\frac{1}{\varepsilon} \Big[ \int_{x}^{x+\varepsilon} f(y) dy \Big] = \\
& = \lim_{\varepsilon \rightarrow 0 } \frac{1}{\varepsilon} \varepsilon f(\xi) = \qquad \xi \in [x,x+\varepsilon] \\
& = \lim_{\varepsilon \rightarrow 0 } f(\xi) = f(x) . \\
\end{aligned}\end{split}
\end{equation*}\label{ch/infinitesimal_calculus/integrals:theorem-2}
\begin{sphinxadmonition}{note}{Theorem 9 (Derivata su dominio dipendente dalla variabile indipendente)}



\sphinxAtStartPar
Sia \(x \in D\), e gli estremi di integrazione \(a(x)\), \(b(x)\) \sphinxstylestrong{todo} \sphinxstyleemphasis{Caratteristiche?}
\begin{equation*}
\begin{split}\dfrac{d}{dx} \int_{a(x)}^{b(x)} f(y) \, dy = - a'(x) \, f(a(x)) + b'(x) f(b(x)) \end{split}
\end{equation*}\end{sphinxadmonition}
\subsubsection*{Dimostrazione}
\begin{equation*}
\begin{split}\begin{aligned}
\dfrac{d}{dx} \int_{a(x)}^{b(x)} f(y) dy & = \lim_{\varepsilon \rightarrow 0 }\frac{1}{\varepsilon} \Big[ \int_{a(x+\varepsilon)}^{b(x+\varepsilon)} f(y) dy - \int_{a(x)}^{b(x)} f(y) dy \Big] = \\
& = \lim_{\varepsilon \rightarrow 0 } \frac{1}{\varepsilon} \Big[ \int_{a(x)}^{b(x)} f(y) dy - \int_{a(x)}^{a(x+\varepsilon)} f(y) dy + \int_{b(x)}^{b(x+\varepsilon)} f(y) dy -  \int_{a(x)}^{b(x)} f(y) dy  \Big] = \\
& = \lim_{\varepsilon \rightarrow 0 } \frac{1}{\varepsilon} \Big[ - \int_{a(x)}^{a(x+\varepsilon)} f(y) dy + \int_{b(x)}^{b(x+\varepsilon)} f(y) dy \Big] = \\
& = \lim_{\varepsilon \rightarrow 0 } \frac{1}{\varepsilon} \Big[ - ( a(x+\varepsilon) - a(x) ) f(\alpha) + ( b(x+\varepsilon) - b(x) ) f(\beta) \Big] = \qquad \alpha \in [a(x), a(x+\varepsilon)] \ , \quad \beta \in [b(x), b(x+\varepsilon)] \\
& = \lim_{\varepsilon \rightarrow 0 } \frac{1}{\varepsilon} \Big[ - ( \varepsilon a'(x) + o(\varepsilon) ) f(\alpha) + ( \varepsilon b'(x) + o(\varepsilon) ) f(\beta) \Big] = \\
& = \lim_{\varepsilon \rightarrow 0 } \frac{1}{\varepsilon} \Big[ - \varepsilon a'(x) \, f(\alpha) + \varepsilon b'(x) \, f(\beta) \Big] =  \\
& = \lim_{\varepsilon \rightarrow 0 } \Big[ - a'(x) \, f(\alpha) +  b'(x) \, f(\beta) \Big] =  \\
& =  - a'(x) \, f(a(x)) +  b'(x) \, f(b(x))  \ .
\end{aligned}\end{split}
\end{equation*}

\section{Integrali fondamentali}
\label{\detokenize{ch/infinitesimal_calculus/integrals:integrali-fondamentali}}\label{\detokenize{ch/infinitesimal_calculus/integrals:infinitesimal-calculus-integrals-fund}}
\sphinxAtStartPar
Una volta dimostrato il {\hyperref[\detokenize{ch/infinitesimal_calculus/integrals:infinitesimal-calculus-derivatives-thm-fund}]{\sphinxcrossref{\DUrole{std,std-ref}{teorema fondamentale del calcolo infinitesimale}}}}, questo risultato può essere usato per valutare gli integrali fondamentali come l’operazione inversa alla derivazione applicata alle {\hyperref[\detokenize{ch/infinitesimal_calculus/derivatives:infinitesimal-calculus-derivatives-fund}]{\sphinxcrossref{\DUrole{std,std-ref}{derivate fondamentali}}}}
\begin{equation*}
\begin{split}\begin{aligned}
 \int x^n         \ dx & = \frac{1}{n} x^{n+1} + C  \qquad (n \neq 0) \\ 
 \int e^x         \ dx & = e^x                 + C \\ 
 \int \frac{1}{x} \ dx & = \ln x               + C \\ 
 \int \cos x      \ dx & = \sin x              + C \\ 
 \int \sin x      \ dx & =-\cos x              + C    
\end{aligned}\end{split}
\end{equation*}

\section{Regole di integrazione}
\label{\detokenize{ch/infinitesimal_calculus/integrals:regole-di-integrazione}}

\subsection{Integrazione per parti}
\label{\detokenize{ch/infinitesimal_calculus/integrals:integrazione-per-parti}}\label{\detokenize{ch/infinitesimal_calculus/integrals:infinitesimal-calculus-integrals-by-parts}}
\sphinxAtStartPar
La regola di integrazione per parti viene ottenuta integrando la regola di {\hyperref[\detokenize{ch/infinitesimal_calculus/derivatives:equation-infinitesimal-calculus-derivatives-rules-product}]{\sphinxcrossref{(4)}}}. Siano \(F(x)\), \(G(x)\) le primitive delle funzioni \(f(x)\), \(g(x)\), e quindi vale \(F'(x) = f(x)\), \(G'(x) = g(x)\).
La regola di derivazione del prodotto \(F(x)G(x)\) viene scritta come
\begin{equation*}
\begin{split}\begin{aligned}
  (F(x) G(x))' & = F'(x) G(x) + F(x) G'(x) = \\
  & = f(x) G(x) + F(x) g(x)
\end{aligned}\end{split}
\end{equation*}
\sphinxAtStartPar
Isolando il termine \(f(x)G(x)\) e integrando, si ottiene
\begin{equation*}
\begin{split}\begin{aligned}
\int f(x) G(x) dx & = \int (F(x) G(x))' dx - \int F(x) g(x) dx = \\
& = F(x) G(x) - \int F(x) g(x) dx  \ .
\end{aligned}\end{split}
\end{equation*}

\subsection{Integrazione con sostituzione}
\label{\detokenize{ch/infinitesimal_calculus/integrals:integrazione-con-sostituzione}}\label{\detokenize{ch/infinitesimal_calculus/integrals:infinitesimal-calculus-integrals-substitution}}
\sphinxAtStartPar
La regola di integrazione per parti viene ottenuta dalla regola di {\hyperref[\detokenize{ch/infinitesimal_calculus/derivatives:equation-infinitesimal-calculus-derivatives-rules-product}]{\sphinxcrossref{(4)}}}. Sia \(\widetilde{F}(x)\) la funzione composta \(\widetilde{F}(x) = F( y(x) )\) e siano definite le derivate
\begin{equation*}
\begin{split}\widetilde{f}(x) = \dfrac{d}{dx} \widetilde{F}(x)  \qquad , \qquad
             f (y) = \dfrac{d}{dy}            F (y)\end{split}
\end{equation*}
\sphinxAtStartPar
per la regola di derivazione della funzione composta,
\begin{equation*}
\begin{split}\widetilde{f}(x) := \dfrac{d}{dx} \widetilde{F}(x) = \dfrac{d}{dx} F(y(x)) = 
\dfrac{d F}{d y}(y(x)) \frac{d y}{d x}(x) =: f(y(x)) y'(x) \ .\end{split}
\end{equation*}
\sphinxAtStartPar
Usando il {\hyperref[\detokenize{ch/infinitesimal_calculus/integrals:infinitesimal-calculus-derivatives-thm-fund}]{\sphinxcrossref{\DUrole{std,std-ref}{teorema del calcolo infinitesimale}}}}

\sphinxAtStartPar
\sphinxstylestrong{todo…}

\sphinxstepscope


\section{Tavola degli integrali indefiniti più comuni}
\label{\detokenize{ch/infinitesimal_calculus/integrals-table:tavola-degli-integrali-indefiniti-piu-comuni}}\label{\detokenize{ch/infinitesimal_calculus/integrals-table:infinitesimal-calculus-integrals-table}}\label{\detokenize{ch/infinitesimal_calculus/integrals-table::doc}}
\sphinxAtStartPar
In questa sezione vengono elencati alcuni tra gli integrali più comuni, la cui valutazione viene lasciata come esercizio, a volte svolto
\begin{equation*}
\begin{split}\int dx             = x + C\end{split}
\end{equation*}\begin{equation*}
\begin{split}\int x^a dx         = \frac{1}{a+1} x^{a+1} + C  \qquad \text{ per } a \ne -1\end{split}
\end{equation*}\begin{equation*}
\begin{split}\int \frac{1}{x} dx = \ln |x| + C  \qquad \text{ per } a \ne -1\end{split}
\end{equation*}\begin{equation*}
\begin{split}\int \frac{1}{1+x^2} dx = \text{atan} x + C\end{split}
\end{equation*}\begin{equation*}
\begin{split}\int \ln x \, dx    = x \, \ln x - x + C\end{split}
\end{equation*}\begin{equation*}
\begin{split}\int \log_b x \, dx    = x \, \log_b x - x \log_b e + C\end{split}
\end{equation*}\begin{equation*}
\begin{split}\int e^{a x} \, dx  = \frac{e^{ax}}{a} + C\end{split}
\end{equation*}\begin{equation*}
\begin{split}\int a^x \, dx  = \frac{a^x}{\ln a} + C\end{split}
\end{equation*}\begin{equation*}
\begin{split}\int \cos x \, dx = \sin x + C\end{split}
\end{equation*}\begin{equation*}
\begin{split}\int \sin x \, dx =-\cos x + C\end{split}
\end{equation*}\begin{equation*}
\begin{split}\int \tan x \, dx = \dots + C\end{split}
\end{equation*}\begin{equation*}
\begin{split}\int \cosh x \, dx = \sinh x + C\end{split}
\end{equation*}\begin{equation*}
\begin{split}\int \sinh x \, dx = \cosh x + C\end{split}
\end{equation*}\begin{equation*}
\begin{split}\int \tanh x \, dx = \dots + C\end{split}
\end{equation*}
\sphinxstepscope


\section{Problemi}
\label{\detokenize{ch/infinitesimal_calculus/integrals-problems:problemi}}\label{\detokenize{ch/infinitesimal_calculus/integrals-problems:infinitesimal-calculus-integrals-problems}}\label{\detokenize{ch/infinitesimal_calculus/integrals-problems::doc}}

\subsection{Calcolo integrali indefiniti}
\label{\detokenize{ch/infinitesimal_calculus/integrals-problems:calcolo-integrali-indefiniti}}\begin{equation*}
\begin{split}\int \dfrac{f'(x)}{f(x)} dx\end{split}
\end{equation*}\begin{equation*}
\begin{split}\int \frac{\sin x}{\cos^2 x} dx\end{split}
\end{equation*}\begin{equation*}
\begin{split}\int \dfrac{f'(x)}{f(x)} dx\end{split}
\end{equation*}\begin{equation*}
\begin{split}\int \frac{1}{a x^2 + b x + c} dx \qquad \text{con } \Delta := b^2 - 4 bc > 0\end{split}
\end{equation*}\begin{equation*}
\begin{split}\int \frac{1}{a x^2 + b x + c} dx \qquad \text{con } \Delta := b^2 - 4 bc < 0\end{split}
\end{equation*}\begin{equation*}
\begin{split}\int f'(x) e^{f(x)} \, dx  = e^{f(x)} + C\end{split}
\end{equation*}\begin{equation*}
\begin{split}\int f'(x) a^{f(x)} \, dx  = \frac{a^{f(x)}}{\ln a} + C\end{split}
\end{equation*}\begin{equation*}
\begin{split}\int f'(x) \, \cos f(x) \, dx = \sin f(x) + C\end{split}
\end{equation*}\begin{equation*}
\begin{split}\int f'(x) \, \sin f(x) \, dx =-\cos f(x) + C\end{split}
\end{equation*}\begin{equation*}
\begin{split}\int \sin^2 x \, dx = \dots\end{split}
\end{equation*}\begin{equation*}
\begin{split}\int \cos^2 x \, dx = \dots\end{split}
\end{equation*}
\sphinxstepscope


\part{Equazioni differenziali ordinarie}

\sphinxstepscope

\begin{sphinxuseclass}{sd-container-fluid}
\begin{sphinxuseclass}{sd-sphinx-override}
\begin{sphinxuseclass}{sd-p-0}
\begin{sphinxuseclass}{sd-mt-2}
\begin{sphinxuseclass}{sd-mb-4}
\begin{sphinxuseclass}{sd-row}
\begin{sphinxuseclass}{sd-row-cols-2}
\begin{sphinxuseclass}{sd-gx-2}
\begin{sphinxuseclass}{sd-gy-1}
\begin{sphinxuseclass}{sd-col}
\begin{sphinxuseclass}{sd-d-flex-row}
\begin{sphinxuseclass}{sd-align-minor-center}
\begin{sphinxuseclass}{sd-container-fluid}
\begin{sphinxuseclass}{sd-sphinx-override}
\begin{sphinxuseclass}{sd-row}
\begin{sphinxuseclass}{sd-row-cols-2}
\begin{sphinxuseclass}{sd-row-cols-xs-2}
\begin{sphinxuseclass}{sd-row-cols-sm-3}
\begin{sphinxuseclass}{sd-row-cols-md-3}
\begin{sphinxuseclass}{sd-row-cols-lg-3}
\begin{sphinxuseclass}{sd-gx-3}
\begin{sphinxuseclass}{sd-gy-1}
\begin{sphinxuseclass}{sd-col}
\begin{sphinxuseclass}{sd-col-auto}
\begin{sphinxuseclass}{sd-d-flex-row}
\begin{sphinxuseclass}{sd-align-minor-center}
\sphinxAtStartPar
basics

\end{sphinxuseclass}
\end{sphinxuseclass}
\end{sphinxuseclass}
\end{sphinxuseclass}
\begin{sphinxuseclass}{sd-col}
\begin{sphinxuseclass}{sd-col-auto}
\begin{sphinxuseclass}{sd-d-flex-row}
\begin{sphinxuseclass}{sd-align-minor-center}
\sphinxAtStartPar
Oct 22, 2024

\end{sphinxuseclass}
\end{sphinxuseclass}
\end{sphinxuseclass}
\end{sphinxuseclass}
\begin{sphinxuseclass}{sd-col}
\begin{sphinxuseclass}{sd-col-auto}
\begin{sphinxuseclass}{sd-d-flex-row}
\begin{sphinxuseclass}{sd-align-minor-center}
\sphinxAtStartPar
2 min read

\end{sphinxuseclass}
\end{sphinxuseclass}
\end{sphinxuseclass}
\end{sphinxuseclass}
\end{sphinxuseclass}
\end{sphinxuseclass}
\end{sphinxuseclass}
\end{sphinxuseclass}
\end{sphinxuseclass}
\end{sphinxuseclass}
\end{sphinxuseclass}
\end{sphinxuseclass}
\end{sphinxuseclass}
\end{sphinxuseclass}
\end{sphinxuseclass}
\end{sphinxuseclass}
\end{sphinxuseclass}
\end{sphinxuseclass}
\end{sphinxuseclass}
\end{sphinxuseclass}
\end{sphinxuseclass}
\end{sphinxuseclass}
\end{sphinxuseclass}
\end{sphinxuseclass}
\end{sphinxuseclass}
\end{sphinxuseclass}

\chapter{Equazioni differenziali ordinarie}
\label{\detokenize{ch/ode:equazioni-differenziali-ordinarie}}\label{\detokenize{ch/ode:ode-high-school}}\label{\detokenize{ch/ode::doc}}
\sphinxAtStartPar
\sphinxstylestrong{Motivazione.} In molti ambiti delle scienze compaiono equazioni differenziali ordinarie, equazioni che impongono una condizione tra una funzione reale incognita e le sue derivate. Così, ad esempio:
\begin{itemize}
\item {} 
\sphinxAtStartPar
le equazioni del moto in dinamica

\item {} 
\sphinxAtStartPar
le equazioni della statica in meccanica delle strutture

\item {} 
\sphinxAtStartPar
le equazioni che descrivono l’andamento della temperatura attraverso un mezzo, in condizioni stazionarie

\item {} 
\sphinxAtStartPar
…
e in generale, in tutti i problemi in cui \sphinxstylestrong{todo}

\end{itemize}

\sphinxAtStartPar
\sphinxstylestrong{Approccio.}
Mentre le motivazioni date dovrebbero essere sufficienti a convincere dell’importanza e della necessità di un’introduzione alle equazioni differenziali ordinarie, una trattazione completa dell’argomento richiede strumenti matematici più avanzati di quelli disponibili a uno studente delle scuole superiori (e spesso anche di molti studenti universitari).

\sphinxAtStartPar
Si cercherà quindi di trattare l’argomento nella maniera più rigorosa possibile per fornire gli strumenti necessari per (semplici) applicazioni nelle quali compaiono le ODE, mentre si chiederà qualche atto di fede nell’accettare alcuni risultati. Per completezza, in corrispondenza di questi atti di fede, verrà messo a disposizione un collegamento a una trattazione più completa dell’argomento.

\sphinxAtStartPar
\sphinxstylestrong{Definizioni.}
Un’equazione differenziale ordinaria è una funzione che coinvolge una funzione reale di una variabile reale, incognita, e le sue derivate. Formalmente una ODE può essere scritta come
\begin{equation*}
\begin{split}F(y^{(n)}(x), \dots y'(x), y(x), x) = 0 \quad , \qquad x \in [x_0, x_1]\end{split}
\end{equation*}
\sphinxAtStartPar
Il \sphinxstylestrong{grado} di una ODE è l’ordine massimo della derivata che compare nell’equazione.

\sphinxAtStartPar
In generale, la soluzione di una ODE di grado \(n\) è il risultato di \(n\) operazioni di integrazione che producono \(n\) costanti arbitrarie. Affinché un problema sia definito, sono necessarie \(n\) condizioni sulla funzione incognita o sulle sue derivate.

\sphinxAtStartPar
Si possono definire alcuni problemi:
\begin{itemize}
\item {} 
\sphinxAtStartPar
problemi differenziali ai valori iniziali

\item {} 
\sphinxAtStartPar
problemi differenziali con condizioni al contorno

\end{itemize}

\sphinxAtStartPar
\sphinxstylestrong{Equazioni lineari a coefficienti costanti.}
\begin{itemize}
\item {} 
\sphinxAtStartPar
Soluzione generale (senza dimostrazione): \(y(x) = y_o(x) + y_p(x)\)

\item {} 
\sphinxAtStartPar
Equazioni di primo grado

\end{itemize}
\begin{equation*}
\begin{split}m \dot{x} + c x  = f(t)\end{split}
\end{equation*}
\sphinxAtStartPar
Soluzione dell’equazione omogenea
\begin{equation*}
\begin{split}x_o(t) = C e^{-\frac{c}{m} t}\end{split}
\end{equation*}\begin{itemize}
\item {} 
\sphinxAtStartPar
Equazioni di secondo grado

\end{itemize}
\begin{equation*}
\begin{split}m \ddot{x} + c \dot{x} + k x = f(t)\end{split}
\end{equation*}
\sphinxAtStartPar
Soluzione dell’equazione omogenea
\begin{equation*}
\begin{split}s_{1,2} = \sigma \mp i \omega\end{split}
\end{equation*}\begin{equation*}
\begin{split}\begin{aligned}
x_o(t) & = C_1 e^{s_1 t } + C_2 e^{s_2 t} = \\
       & = e^{\sigma t} \left( C_1 e^{-i \omega t } + C_2 e^{i \omega t} \right) \ ,
\end{aligned}\end{split}
\end{equation*}
\sphinxAtStartPar
con le costanti di integrazione complesse coniugate,
\begin{equation*}
\begin{split}C_1 = C_2^* = (A - i B)^* = A + i B\end{split}
\end{equation*}
\sphinxAtStartPar
al fine di avere una soluzione reale. Ricordando che la somma di un numero complesso e del suo coniugato vale due volte la parte reale,
\begin{equation*}
\begin{split}w + w^* = (u+iv) + (u+iv)^* = u+iv + u - i v = 2 u = 2 \text{Re}\{w\}\end{split}
\end{equation*}
\sphinxAtStartPar
si può riscrivere la soluzione dell’equazione omogenea
\begin{equation*}
\begin{split}x_o(t) = 2 \left[ A \cos ( \omega t ) + B \sin (\omega t ) \right] \end{split}
\end{equation*}
\sphinxAtStartPar
avendo riconosciuto
\begin{equation*}
\begin{split}\begin{aligned}
  \text{Re}\{C_2 e^{i \omega t}\} &= \text{Re}\{ (A - i B)(\cos(\omega t) + i \sin(\omega t)\} = \\
  & = \text{Re}\{ A \cos(\omega t) + B \sin (\omega t) + i \left[ A \sin (\omega t) - B \cos (\omega t) \right]\} \ .
\end{aligned}\end{split}
\end{equation*}
\sphinxAtStartPar
\sphinxstylestrong{Equazioni separabili: tecnica di soluzione di separazione delle variabili.}
\begin{equation*}
\begin{split}\frac{d y}{d x} = f(y(x)) \ g(x) \end{split}
\end{equation*}
\sphinxAtStartPar
può essere riscritta formalmente come
\begin{equation*}
\begin{split}\dfrac{dy}{f(y)} = g(x) \ d x \end{split}
\end{equation*}
\sphinxAtStartPar
e integrata con le opportune condizioni
\begin{equation*}
\begin{split}\tilde{F}(y(x)) - \tilde{F}(y(x_0)) = G(x) - G(x_0)\end{split}
\end{equation*}
\sphinxAtStartPar
\sphinxstylestrong{Esempi}
\begin{itemize}
\item {} 
\sphinxAtStartPar
Moto rettilineo in un campo di forze costante e uniforme

\end{itemize}
\begin{equation*}
\begin{split}m \ddot{x} = f =: m g\end{split}
\end{equation*}
\sphinxAtStartPar
L’integrazione produce
\begin{equation*}
\begin{split}x(t) = \frac{1}{2} g t^2 + v_0 t + x_0\end{split}
\end{equation*}\begin{itemize}
\item {} 
\sphinxAtStartPar
Moto di un corpo in un campo di forze costante e uniforme e forza viscosa (lineare e quadratica)

\end{itemize}
\begin{equation*}
\begin{split}\begin{cases}
m \dot{v} = f + f^{visc} \\
\dot{x} = v
\end{cases}\end{split}
\end{equation*}
\sphinxAtStartPar
\sphinxstylestrong{resistenza proporzionale alla velocità}
\$\(\begin{cases}
m \dot{v} + c v = mg \\
\dot{x} = v
\end{cases}\)\$

\sphinxAtStartPar
\sphinxstylestrong{resistenza proporzionale al quadrato della velocità}
\$\(\begin{cases}
m \dot{v} + \frac{1}{2} \rho S c_x v^2 = mg \\
\dot{x} = v
\end{cases}\)\$
\begin{itemize}
\item {} 
\sphinxAtStartPar
Temperatura della testa di una termocoppia

\end{itemize}
\begin{equation*}
\begin{split}\dot{E} = \dot{Q} \ ,\end{split}
\end{equation*}
\sphinxAtStartPar
con \(E = m c T\), \(\dot{Q} = h (T_e - T)\)
\begin{equation*}
\begin{split}m c \dot{T} + h T = h T_e\end{split}
\end{equation*}\begin{itemize}
\item {} 
\sphinxAtStartPar
Distribuzione della temperatura, in un caso stazionario

\end{itemize}
\begin{equation*}
\begin{split}(k T')' = \rho r\end{split}
\end{equation*}\begin{itemize}
\item {} 
\sphinxAtStartPar
Sistema massa\sphinxhyphen{}molla\sphinxhyphen{}smorzatore, libero e forzato (\sphinxstylestrong{todo} risonanza)

\end{itemize}
\begin{equation*}
\begin{split}m \ddot{x} + c \dot{x} + k x = f\end{split}
\end{equation*}\begin{itemize}
\item {} 
\sphinxAtStartPar
Circuiti RLC (analogia formale con sistema MMS)

\end{itemize}

\sphinxAtStartPar
\(\dots\)
\begin{itemize}
\item {} 
\sphinxAtStartPar
Deformazione a trazione di una trave

\end{itemize}
\begin{equation*}
\begin{split}(EA u')' = f\end{split}
\end{equation*}\begin{itemize}
\item {} 
\sphinxAtStartPar
Deformazione a flessione di una trave

\end{itemize}
\begin{equation*}
\begin{split}(EJ w'')'' = f\end{split}
\end{equation*}
\sphinxstepscope


\part{Trigonometria}

\sphinxstepscope


\chapter{Trigonometria}
\label{\detokenize{ch/trigonometry:trigonometria}}\label{\detokenize{ch/trigonometry:geometry-trigonometry}}\label{\detokenize{ch/trigonometry::doc}}

\section{Prime definizioni e relazioni}
\label{\detokenize{ch/trigonometry:prime-definizioni-e-relazioni}}\begin{equation*}
\begin{split}\sin \theta = \frac{\overline{PH}}{R}\end{split}
\end{equation*}\begin{equation*}
\begin{split}\cos \theta = \frac{\overline{OH}}{R}\end{split}
\end{equation*}
\sphinxAtStartPar
Usando il teorema di Pitagora è immediato dimostrare la relazione fondamentale tra le funzioni trigonometriche
\begin{equation*}
\begin{split}\sin^2 \theta + \cos^2 \theta = 1 \ .\end{split}
\end{equation*}
\sphinxAtStartPar
\sphinxstylestrong{Nota sulla notazione.} Nell’uso delle funzioni trigonometriche, \(\sin^2 x\) indica il quadrato della funzione e non la composizione della funzione con se stessa,
\begin{equation*}
\begin{split}\sin^2 x = (\sin x)^2 \neq $\sin( \sin x) \ .\end{split}
\end{equation*}
\sphinxAtStartPar
\sphinxstylestrong{Tangente.} \$\(\tan \theta = \dfrac{\sin \theta}{\cos \theta} \ .\)\$

\sphinxAtStartPar
\sphinxstyleemphasis{Cosecante, secante, cotangente.} Definizioni al limite dell’inutile e dannoso…


\section{Proprietà}
\label{\detokenize{ch/trigonometry:proprieta}}\begin{equation*}
\begin{split}\sin\left(x+\frac{\pi}{2}\right) = \cos x\end{split}
\end{equation*}\begin{equation*}
\begin{split}\cos\left(x+\frac{\pi}{2}\right) =-\sin x\end{split}
\end{equation*}
\sphinxAtStartPar
…
\$\(\sin(\pi - x) = \sin x\)\(
\)\(\cos(\pi - x) =-\cos x\)\$
…


\section{Angoli particolari}
\label{\detokenize{ch/trigonometry:angoli-particolari}}\begin{equation*}
\begin{split}\begin{aligned}
\cos 0 = 1                              \qquad & , \qquad \sin 0 = 0 \\
\cos \frac{\pi}{6} = \frac{\sqrt{3}}{2} \qquad & , \qquad \sin \frac{\pi}{6} = \frac{1}{2}        \\
\cos \frac{\pi}{4} = \frac{\sqrt{2}}{2} \qquad & , \qquad \sin \frac{\pi}{6} = \frac{\sqrt{2}}{2} \\
\cos \frac{\pi}{3} = \frac{1}{2}        \qquad & , \qquad \sin \frac{\pi}{6} = \frac{\sqrt{3}}{2} \\
\cos \frac{\pi}{2} = 0                  \qquad & , \qquad \sin \frac{\pi}{6} = 1 \\
\end{aligned}\end{split}
\end{equation*}

\section{Formule di somma e sottrazione}
\label{\detokenize{ch/trigonometry:formule-di-somma-e-sottrazione}}\begin{equation*}
\begin{split}\cos ( x \mp y ) = \cos x \ \cos y \pm \sin x \ \sin y\end{split}
\end{equation*}\begin{equation*}
\begin{split}\sin ( x \mp y ) = \sin x \ \cos y \mp \cos x \ \sin y\end{split}
\end{equation*}
\sphinxAtStartPar
\sphinxstylestrong{todo} dimostrazione con geometria
\begin{equation*}
\begin{split}\cos ( \alpha + \beta ) = \frac{OF}{R}\end{split}
\end{equation*}\begin{equation*}
\begin{split}\sin ( \alpha + \beta ) = \frac{BF}{R}\end{split}
\end{equation*}\begin{equation*}
\begin{split}OF = \frac{OC}{OA} OE = \cos \alpha \ OE\end{split}
\end{equation*}\begin{equation*}
\begin{split}OE = OD - ED = OB \cos \beta - OB \sin \beta \frac{\sin \alpha}{\cos \alpha}\end{split}
\end{equation*}\begin{equation*}
\begin{split}OF = OB \left( \cos \alpha \cos \beta - \sin \beta \sin \alpha  \right)\end{split}
\end{equation*}\begin{equation*}
\begin{split}\cos (\alpha + \beta) = \cos \alpha \cos \beta - \sin \beta \sin \alpha\end{split}
\end{equation*}

\section{Werner}
\label{\detokenize{ch/trigonometry:werner}}\begin{equation*}
\begin{split}\cos x \cos y = \dfrac{1}{2} \left[ \cos( x - y ) + \cos ( x + y ) \right]\end{split}
\end{equation*}\begin{equation*}
\begin{split}\sin x \sin y = \dfrac{1}{2} \left[ \cos( x - y ) - \cos ( x + y ) \right]\end{split}
\end{equation*}\begin{equation*}
\begin{split}\sin x \cos y = \dfrac{1}{2} \left[ \sin( x - y ) + \sin ( x + y ) \right]\end{split}
\end{equation*}

\section{Prostaferesi}
\label{\detokenize{ch/trigonometry:prostaferesi}}
\sphinxAtStartPar
Definendo \(p = x-y\) e \(q = x+y\) nelle formule di Werner, è immediato ricavare
\begin{equation*}
\begin{split}\cos p + \cos q = 2 \cos\left(\frac{p+q}{2} \right) \cos\left(\frac{q-p}{2} \right)\end{split}
\end{equation*}\begin{equation*}
\begin{split}\cos p - \cos q = 2 \sin\left(\frac{p+q}{2} \right) \sin\left(\frac{q-p}{2} \right)\end{split}
\end{equation*}\begin{equation*}
\begin{split}\sin p + \sin q = 2 \sin\left(\frac{p+q}{2} \right) \cos\left(\frac{q-p}{2} \right)\end{split}
\end{equation*}
\sphinxstepscope


\part{Esponenziale e logaritmo}

\sphinxstepscope


\chapter{Esponenziale e logaritmo}
\label{\detokenize{ch/exponential_logarithm:esponenziale-e-logaritmo}}\label{\detokenize{ch/exponential_logarithm:geometry-exponential-logarithm}}\label{\detokenize{ch/exponential_logarithm::doc}}

\section{Definizioni e proprietà}
\label{\detokenize{ch/exponential_logarithm:definizioni-e-proprieta}}
\sphinxAtStartPar
Nel campo reale, per ogni \(b > 0\),
\begin{equation*}
\begin{split}a = b^c \qquad \leftrightarrow \qquad c = \log_{b} a \end{split}
\end{equation*}

\section{Funzione esponenziale e logaritmo}
\label{\detokenize{ch/exponential_logarithm:funzione-esponenziale-e-logaritmo}}

\section{\protect\(e\protect\) di Nepero, \protect\(e^x\protect\) e logaritmo naturale}
\label{\detokenize{ch/exponential_logarithm:e-di-nepero-e-x-e-logaritmo-naturale}}

\subsection{Definizione di \protect\(e^x\protect\)}
\label{\detokenize{ch/exponential_logarithm:definizione-di-e-x}}
\sphinxAtStartPar
\sphinxstylestrong{Esponenziale reale.} Per ogni \(x \in \mathbb{R}\), si definisce la funzione \(e^x\) come
\begin{equation*}
\begin{split}\begin{aligned}
e^x & := \lim_{n \rightarrow \infty} \left( 1 + \frac{x}{n}\right)^n  = \\
    & := \lim_{n \rightarrow \infty} \sum_{k=0}^{n} \frac{x^n}{n!} \\
\end{aligned}\end{split}
\end{equation*}
\sphinxAtStartPar
Si può dimostrare che
\begin{itemize}
\item {} 
\sphinxAtStartPar
le due definizioni sono equivalenti, e la serie è convergente per ogni \(\mathbf{x} \in \mathbb{R}\) finito

\item {} 
\sphinxAtStartPar
la funzione \(e^x\) giustifica questa notazione poiché soddisfa le proprietà delle potenze, come
\begin{equation*}
\begin{split}e^{x+y} = e^x \, e^y \ .\end{split}
\end{equation*}
\item {} 
\sphinxAtStartPar
la base della potenza, \(e\), viene definita \(e\) \sphinxstylestrong{di Nepero}, ed è un numero reale irrazionale, il cui valore approssimato è \(e \approx 2.718281828\text{"e poi la magia finisce"}\): nonostante le prime cifre decimali facciano pensare che possa essere periodico, se si scrivono le cifre successive, l’approssimazione diventa \(e \approx 2.71828182845904523\dots\)

\end{itemize}

\sphinxAtStartPar
\sphinxstylestrong{Esponenziale complesso.} Si può estendere la definizione di esponenziale anche a un numero complesso, \(z \in \mathbb{C}\)
\sphinxstylestrong{todo}

\sphinxstepscope


\part{Serie e successioni}

\sphinxstepscope

\begin{sphinxuseclass}{sd-container-fluid}
\begin{sphinxuseclass}{sd-sphinx-override}
\begin{sphinxuseclass}{sd-p-0}
\begin{sphinxuseclass}{sd-mt-2}
\begin{sphinxuseclass}{sd-mb-4}
\begin{sphinxuseclass}{sd-row}
\begin{sphinxuseclass}{sd-row-cols-2}
\begin{sphinxuseclass}{sd-gx-2}
\begin{sphinxuseclass}{sd-gy-1}
\begin{sphinxuseclass}{sd-col}
\begin{sphinxuseclass}{sd-d-flex-row}
\begin{sphinxuseclass}{sd-align-minor-center}
\begin{sphinxuseclass}{sd-container-fluid}
\begin{sphinxuseclass}{sd-sphinx-override}
\begin{sphinxuseclass}{sd-row}
\begin{sphinxuseclass}{sd-row-cols-2}
\begin{sphinxuseclass}{sd-row-cols-xs-2}
\begin{sphinxuseclass}{sd-row-cols-sm-3}
\begin{sphinxuseclass}{sd-row-cols-md-3}
\begin{sphinxuseclass}{sd-row-cols-lg-3}
\begin{sphinxuseclass}{sd-gx-3}
\begin{sphinxuseclass}{sd-gy-1}
\begin{sphinxuseclass}{sd-col}
\begin{sphinxuseclass}{sd-col-auto}
\begin{sphinxuseclass}{sd-d-flex-row}
\begin{sphinxuseclass}{sd-align-minor-center}
\sphinxAtStartPar
basics

\end{sphinxuseclass}
\end{sphinxuseclass}
\end{sphinxuseclass}
\end{sphinxuseclass}
\begin{sphinxuseclass}{sd-col}
\begin{sphinxuseclass}{sd-col-auto}
\begin{sphinxuseclass}{sd-d-flex-row}
\begin{sphinxuseclass}{sd-align-minor-center}
\sphinxAtStartPar
Oct 22, 2024

\end{sphinxuseclass}
\end{sphinxuseclass}
\end{sphinxuseclass}
\end{sphinxuseclass}
\begin{sphinxuseclass}{sd-col}
\begin{sphinxuseclass}{sd-col-auto}
\begin{sphinxuseclass}{sd-d-flex-row}
\begin{sphinxuseclass}{sd-align-minor-center}
\sphinxAtStartPar
0 min read

\end{sphinxuseclass}
\end{sphinxuseclass}
\end{sphinxuseclass}
\end{sphinxuseclass}
\end{sphinxuseclass}
\end{sphinxuseclass}
\end{sphinxuseclass}
\end{sphinxuseclass}
\end{sphinxuseclass}
\end{sphinxuseclass}
\end{sphinxuseclass}
\end{sphinxuseclass}
\end{sphinxuseclass}
\end{sphinxuseclass}
\end{sphinxuseclass}
\end{sphinxuseclass}
\end{sphinxuseclass}
\end{sphinxuseclass}
\end{sphinxuseclass}
\end{sphinxuseclass}
\end{sphinxuseclass}
\end{sphinxuseclass}
\end{sphinxuseclass}
\end{sphinxuseclass}
\end{sphinxuseclass}
\end{sphinxuseclass}

\chapter{Serie e successioni}
\label{\detokenize{ch/series:serie-e-successioni}}\label{\detokenize{ch/series:series-high-school}}\label{\detokenize{ch/series::doc}}
\sphinxstepscope


\part{Numeri complessi}

\sphinxstepscope

\begin{sphinxuseclass}{sd-container-fluid}
\begin{sphinxuseclass}{sd-sphinx-override}
\begin{sphinxuseclass}{sd-p-0}
\begin{sphinxuseclass}{sd-mt-2}
\begin{sphinxuseclass}{sd-mb-4}
\begin{sphinxuseclass}{sd-row}
\begin{sphinxuseclass}{sd-row-cols-2}
\begin{sphinxuseclass}{sd-gx-2}
\begin{sphinxuseclass}{sd-gy-1}
\begin{sphinxuseclass}{sd-col}
\begin{sphinxuseclass}{sd-d-flex-row}
\begin{sphinxuseclass}{sd-align-minor-center}
\begin{sphinxuseclass}{sd-container-fluid}
\begin{sphinxuseclass}{sd-sphinx-override}
\begin{sphinxuseclass}{sd-row}
\begin{sphinxuseclass}{sd-row-cols-2}
\begin{sphinxuseclass}{sd-row-cols-xs-2}
\begin{sphinxuseclass}{sd-row-cols-sm-3}
\begin{sphinxuseclass}{sd-row-cols-md-3}
\begin{sphinxuseclass}{sd-row-cols-lg-3}
\begin{sphinxuseclass}{sd-gx-3}
\begin{sphinxuseclass}{sd-gy-1}
\begin{sphinxuseclass}{sd-col}
\begin{sphinxuseclass}{sd-col-auto}
\begin{sphinxuseclass}{sd-d-flex-row}
\begin{sphinxuseclass}{sd-align-minor-center}
\sphinxAtStartPar
basics

\end{sphinxuseclass}
\end{sphinxuseclass}
\end{sphinxuseclass}
\end{sphinxuseclass}
\begin{sphinxuseclass}{sd-col}
\begin{sphinxuseclass}{sd-col-auto}
\begin{sphinxuseclass}{sd-d-flex-row}
\begin{sphinxuseclass}{sd-align-minor-center}
\sphinxAtStartPar
Oct 22, 2024

\end{sphinxuseclass}
\end{sphinxuseclass}
\end{sphinxuseclass}
\end{sphinxuseclass}
\begin{sphinxuseclass}{sd-col}
\begin{sphinxuseclass}{sd-col-auto}
\begin{sphinxuseclass}{sd-d-flex-row}
\begin{sphinxuseclass}{sd-align-minor-center}
\sphinxAtStartPar
1 min read

\end{sphinxuseclass}
\end{sphinxuseclass}
\end{sphinxuseclass}
\end{sphinxuseclass}
\end{sphinxuseclass}
\end{sphinxuseclass}
\end{sphinxuseclass}
\end{sphinxuseclass}
\end{sphinxuseclass}
\end{sphinxuseclass}
\end{sphinxuseclass}
\end{sphinxuseclass}
\end{sphinxuseclass}
\end{sphinxuseclass}
\end{sphinxuseclass}
\end{sphinxuseclass}
\end{sphinxuseclass}
\end{sphinxuseclass}
\end{sphinxuseclass}
\end{sphinxuseclass}
\end{sphinxuseclass}
\end{sphinxuseclass}
\end{sphinxuseclass}
\end{sphinxuseclass}
\end{sphinxuseclass}
\end{sphinxuseclass}

\chapter{Numeri complessi}
\label{\detokenize{ch/complex:numeri-complessi}}\label{\detokenize{ch/complex:complex-numbers-high-school}}\label{\detokenize{ch/complex::doc}}
\sphinxAtStartPar
Utilità dei numeri complessi:
\begin{itemize}
\item {} 
\sphinxAtStartPar
utilizzo in molti ambiti della matematica, della fisica e dell’ingegneria: soluzione ODE, soluzione PDE, teoria delle trasformate

\item {} 
\sphinxAtStartPar
facile rappresentazione di funzioni armoniche, grazie all’identità di Eulero

\end{itemize}

\sphinxAtStartPar
\sphinxstylestrong{Argomenti}
\begin{itemize}
\item {} 
\sphinxAtStartPar
Definizioni e rappresentazioni

\item {} 
\sphinxAtStartPar
Algebra:
\begin{itemize}
\item {} 
\sphinxAtStartPar
operazioni

\item {} 
\sphinxAtStartPar
equazioni e disequazioni

\item {} 
\sphinxAtStartPar
teorema fondamentale dell’algebra

\end{itemize}

\end{itemize}


\section{Definizione}
\label{\detokenize{ch/complex:definizione}}
\sphinxAtStartPar
I numeri complessi estendono il campo dei numeri reali, grazie all’introduzione dell’\sphinxstylestrong{unità immaginaria}, \(i\), definita come la radice quadra di \(-1\),
\begin{equation*}
\begin{split}i := \sqrt{-1} \ .\end{split}
\end{equation*}
\sphinxAtStartPar
L’insieme dei numeri complessi, indicato con \(\mathbb{C}\), è l’insieme di quei numeri che possono essere scritti come
\begin{equation*}
\begin{split}z = x + i y \ ,\end{split}
\end{equation*}
\sphinxAtStartPar
con \(x, \ y \in \mathbb{R}\).

\sphinxAtStartPar
I numeri complessi formano la struttura algebrica di \sphinxstylestrong{campo} con le operazioni di somma e prodotto. Dati due numeri complessi \(z_1 = x_1 + i y_1\), \(z_2 = x_2 + i y_2\), grazie alle proprietà \sphinxstylestrong{todo} delle operazioni, si può scrivere
\begin{itemize}
\item {} 
\sphinxAtStartPar
la somma
\begin{equation*}
\begin{split}z_1 + z_2 = (x_1 + x_2) + i (y_1 + y_2) \ \end{split}
\end{equation*}
\item {} 
\sphinxAtStartPar
il prodotto,

\end{itemize}
\begin{equation*}
\begin{split}\begin{aligned}
z_1 \, z_2 & = (x_1 + i y_1) (x_2 + i y_2) = \\
& = x_1 \, x_2 - y_1 \, y_2 + i ( x_1 \, y_2 + x_2 \, y_1 )
\end{aligned}\end{split}
\end{equation*}

\section{Rappresentazione nel piano complesso}
\label{\detokenize{ch/complex:rappresentazione-nel-piano-complesso}}
\sphinxAtStartPar
Ogni numero complesso \(z = x + i \, y\) può essere rappresentato nel piano complesso, … \sphinxstylestrong{todo}

\sphinxAtStartPar
La rappresentazione grafica suggerisce una rappresentazione alternativa, la \sphinxstylestrong{rappresentazione polare}, con il cambio di coordinate
\begin{equation*}
\begin{split}\begin{cases}
x = r \cos \theta \\
y = r \sin \theta
\end{cases} \quad \rightarrow \quad 
\begin{aligned}
z & = r \cos \theta + i \ r \sin \theta = \\
  & = r \left( \cos \theta + i \sin \theta \right) = r \, e^{i \, \theta} \ ,
\end{aligned}\end{split}
\end{equation*}
\sphinxAtStartPar
avendo anticipato qui la \sphinxstylestrong{formula di Eulero} per l’esponenziale di un numero immaginario,
\begin{equation*}
\begin{split}e^{i \, \theta} = \cos \theta + i \, \sin \theta \ .\end{split}
\end{equation*}
\sphinxAtStartPar
\sphinxstylestrong{todo}
\begin{itemize}
\item {} 
\sphinxAtStartPar
Riferimento alla formula di Eulero. Dimostrazione con serie? Cosa serve? Serie di Taylor? Crietri di convergenza delle serie?

\item {} 
\sphinxAtStartPar
Riferimento alla definizione di esponenziale

\end{itemize}

\sphinxAtStartPar
\sphinxstylestrong{todo} Le due rappresentazioni non sono equivalenti. Mentre la rappresentazione cartesiana permette di creare una relazione biunivoca tra i numeri complessi \(z = x + i \ y\) e i punti nel piano \((x, \ y)\), la rappresentazione polare assegna infiniti numeri complessi, seppur di uguale valore \(r \, e^{i \theta} = r \, e^{i (\theta + n \, 2 \pi)}\), con \(n \in \mathbb{Z}\) allo stesso punto nello spazio.


\section{Operazioni con i numeri complessi}
\label{\detokenize{ch/complex:operazioni-con-i-numeri-complessi}}\begin{itemize}
\item {} 
\sphinxAtStartPar
somma
\begin{equation*}
\begin{split}z_1 + z_2 = (x_1 + x_2) + i(y_1 + y_2)\end{split}
\end{equation*}
\item {} 
\sphinxAtStartPar
prodotto
\begin{equation*}
\begin{split}z_1 \ z_2 = r_1 \, r_2 e^{i(\theta_1 + \theta_2)}\end{split}
\end{equation*}
\item {} 
\sphinxAtStartPar
valore assoluto
\begin{equation*}
\begin{split}|z| = \sqrt{x^2 + y^2} = r\end{split}
\end{equation*}
\item {} 
\sphinxAtStartPar
potenza

\item {} 
\sphinxAtStartPar
esponenziale

\item {} 
\sphinxAtStartPar
logaritmo

\end{itemize}

\sphinxstepscope


\part{Algebra lineare}

\sphinxstepscope

\begin{sphinxuseclass}{sd-container-fluid}
\begin{sphinxuseclass}{sd-sphinx-override}
\begin{sphinxuseclass}{sd-p-0}
\begin{sphinxuseclass}{sd-mt-2}
\begin{sphinxuseclass}{sd-mb-4}
\begin{sphinxuseclass}{sd-row}
\begin{sphinxuseclass}{sd-row-cols-2}
\begin{sphinxuseclass}{sd-gx-2}
\begin{sphinxuseclass}{sd-gy-1}
\begin{sphinxuseclass}{sd-col}
\begin{sphinxuseclass}{sd-d-flex-row}
\begin{sphinxuseclass}{sd-align-minor-center}
\begin{sphinxuseclass}{sd-container-fluid}
\begin{sphinxuseclass}{sd-sphinx-override}
\begin{sphinxuseclass}{sd-row}
\begin{sphinxuseclass}{sd-row-cols-2}
\begin{sphinxuseclass}{sd-row-cols-xs-2}
\begin{sphinxuseclass}{sd-row-cols-sm-3}
\begin{sphinxuseclass}{sd-row-cols-md-3}
\begin{sphinxuseclass}{sd-row-cols-lg-3}
\begin{sphinxuseclass}{sd-gx-3}
\begin{sphinxuseclass}{sd-gy-1}
\begin{sphinxuseclass}{sd-col}
\begin{sphinxuseclass}{sd-col-auto}
\begin{sphinxuseclass}{sd-d-flex-row}
\begin{sphinxuseclass}{sd-align-minor-center}
\sphinxAtStartPar
basics

\end{sphinxuseclass}
\end{sphinxuseclass}
\end{sphinxuseclass}
\end{sphinxuseclass}
\begin{sphinxuseclass}{sd-col}
\begin{sphinxuseclass}{sd-col-auto}
\begin{sphinxuseclass}{sd-d-flex-row}
\begin{sphinxuseclass}{sd-align-minor-center}
\sphinxAtStartPar
Oct 22, 2024

\end{sphinxuseclass}
\end{sphinxuseclass}
\end{sphinxuseclass}
\end{sphinxuseclass}
\begin{sphinxuseclass}{sd-col}
\begin{sphinxuseclass}{sd-col-auto}
\begin{sphinxuseclass}{sd-d-flex-row}
\begin{sphinxuseclass}{sd-align-minor-center}
\sphinxAtStartPar
0 min read

\end{sphinxuseclass}
\end{sphinxuseclass}
\end{sphinxuseclass}
\end{sphinxuseclass}
\end{sphinxuseclass}
\end{sphinxuseclass}
\end{sphinxuseclass}
\end{sphinxuseclass}
\end{sphinxuseclass}
\end{sphinxuseclass}
\end{sphinxuseclass}
\end{sphinxuseclass}
\end{sphinxuseclass}
\end{sphinxuseclass}
\end{sphinxuseclass}
\end{sphinxuseclass}
\end{sphinxuseclass}
\end{sphinxuseclass}
\end{sphinxuseclass}
\end{sphinxuseclass}
\end{sphinxuseclass}
\end{sphinxuseclass}
\end{sphinxuseclass}
\end{sphinxuseclass}
\end{sphinxuseclass}
\end{sphinxuseclass}

\chapter{Algebra lineare}
\label{\detokenize{ch/linear_algebra:algebra-lineare}}\label{\detokenize{ch/linear_algebra:linear-algebra-high-school}}\label{\detokenize{ch/linear_algebra::doc}}





\renewcommand{\indexname}{Proof Index}
\begin{sphinxtheindex}
\let\bigletter\sphinxstyleindexlettergroup
\bigletter{theorem\sphinxhyphen{}0}
\item\relax\sphinxstyleindexentry{theorem\sphinxhyphen{}0}\sphinxstyleindexextra{ch/infinitesimal\_calculus/integrals}\sphinxstyleindexpageref{ch/infinitesimal_calculus/integrals:\detokenize{theorem-0}}
\indexspace
\bigletter{theorem\sphinxhyphen{}1}
\item\relax\sphinxstyleindexentry{theorem\sphinxhyphen{}1}\sphinxstyleindexextra{ch/infinitesimal\_calculus/integrals}\sphinxstyleindexpageref{ch/infinitesimal_calculus/integrals:\detokenize{theorem-1}}
\indexspace
\bigletter{theorem\sphinxhyphen{}2}
\item\relax\sphinxstyleindexentry{theorem\sphinxhyphen{}2}\sphinxstyleindexextra{ch/infinitesimal\_calculus/integrals}\sphinxstyleindexpageref{ch/infinitesimal_calculus/integrals:\detokenize{theorem-2}}
\indexspace
\bigletter{theorem\sphinxhyphen{}3}
\item\relax\sphinxstyleindexentry{theorem\sphinxhyphen{}3}\sphinxstyleindexextra{ch/infinitesimal\_calculus/derivatives}\sphinxstyleindexpageref{ch/infinitesimal_calculus/derivatives:\detokenize{theorem-3}}
\indexspace
\bigletter{theorem\sphinxhyphen{}4}
\item\relax\sphinxstyleindexentry{theorem\sphinxhyphen{}4}\sphinxstyleindexextra{ch/infinitesimal\_calculus/derivatives}\sphinxstyleindexpageref{ch/infinitesimal_calculus/derivatives:\detokenize{theorem-4}}
\indexspace
\bigletter{theorem\sphinxhyphen{}5}
\item\relax\sphinxstyleindexentry{theorem\sphinxhyphen{}5}\sphinxstyleindexextra{ch/infinitesimal\_calculus/derivatives}\sphinxstyleindexpageref{ch/infinitesimal_calculus/derivatives:\detokenize{theorem-5}}
\end{sphinxtheindex}

\renewcommand{\indexname}{Index}
\printindex
\end{document}